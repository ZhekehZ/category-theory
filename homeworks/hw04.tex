\documentclass[draft]{article}
\usepackage[russian]{babel}
\usepackage[utf8]{inputenc}
\usepackage{cmap}
\usepackage{amsfonts}
\usepackage{amsmath}
\usepackage[all]{xy}
\usepackage{bussproofs}

\newcommand{\cat}[1]{\mathbf{#1}}
\renewcommand{\C}{\cat{C}}
\newcommand{\Set}{\cat{Set}}
\newcommand{\Grp}{\cat{Grp}}
\newcommand{\Ab}{\cat{Ab}}
\newcommand{\Hask}{\cat{Hask}}
\newcommand{\Mat}{\cat{Mat}}
\newcommand{\Num}{\cat{Num}}

\newcommand{\im}{\mathrm{Im}}
\newcommand{\bool}{\mathrm{Bool}}
\newcommand{\true}{\mathrm{true}}
\newcommand{\false}{\mathrm{false}}
\newcommand{\andb}{\mathrm{and}}
\newcommand{\orb}{\mathrm{or}}
\newcommand{\inj}{\mathrm{inj}}

\newcommand{\ev}{\mathrm{ev}}
\newcommand{\zero}{\mathrm{zero}}
\newcommand{\suc}{\mathrm{suc}}
\newcommand{\rec}{\mathrm{rec}}

\newenvironment{tolerant}[1]{\par\tolerance=#1\relax}{\par}

\begin{document}

\title{Задания}
\maketitle

\begin{enumerate}

\item При каких условиях в категории (пред)порядка существует булевский объект?

\item Приведите пример нетривиальной категории порядка, являющейся декартово замкнутой.

Пусть объекты --- логические формулы (в какой-нибудь логике). Определим $x \leq y$ как выполнение $x$ влечет выполнение $y$, а произведение как конъюнкцию. Тогда $\forall X, Y: ~~X^Y = XYto X$, где "$\to$"\  есть импликация. $ev = X^Y\times Y \leq X$. Если $X^Y\times Y$ выполнимо, то $X$ выполнимо по Modus ponens. Кроме того, если $\Gamma \land A \to B$, то $\Gamma \to (A \to B)$. Он единственный по построению, а значит экспонента определена верно. 

\item Давайте докажем, что категории моноидов, групп и абелевых групп не являются декартово замкнутыми.
\begin{enumerate}
\item Докажите, что если в декартово замкнутой категории есть начальный объект, то он строгий.
\item Объект называется \emph{нулевым}, если он одновременно начальный и терминальный.
Докажите, что если в категории есть нулевой объект и начальный объект строгий, то эта категория тривиальна (то есть в ней между любой парой объектов существует уникальная стрелка).
\item Докажите, что в категориях, упомянутых в задании, есть нулевой объект и сделайте вывод, что они не декартово замкнуты.
\end{enumerate}
\textit{Решение}
\begin{enumerate}
\item[(c)] Во всех этих категориях есть объект, состоящий из одного нейтрального элемента. Он и будет нулевым. Так как нейтральный должен переходить в нейтральный, то отображение из него в другие объекты единственно. Обратное тоже верно --- все отображается в 0. Очевидно, что 0 не строгий, так как ядро у композиции отображений между 0 и некоторым объектом будет совпадать с тем объектом. Значит эти категории не декартово замкнуты.
\end{enumerate}

\item Пусть в категории $\C$ есть все конечные произведения и булевский объект.
Сконструируйте в $\C$ морфизмы $\andb, \orb : \bool \times \bool \to \bool$, такие что следующие диаграммы коммутируют
\[ \xymatrix{ 1 \ar[d]_{\langle \true, \true \rangle} \ar[rd]^{\true} \\
              \bool \times \bool \ar[r]_-{\andb} & \bool
            }
\qquad \xymatrix{ 1 \ar[d]_{\langle \true, \true \rangle} \ar[rd]^{\true} \\
              \bool \times \bool \ar[r]_-{\orb} & \bool
            } \]
\[ \xymatrix{ 1 \ar[d]_{\langle \true, \false \rangle} \ar[rd]^{\false} \\
              \bool \times \bool \ar[r]_-{\andb} & \bool
            }
\qquad \xymatrix{ 1 \ar[d]_{\langle \true, \false \rangle} \ar[rd]^{\true} \\
              \bool \times \bool \ar[r]_-{\orb} & \bool
            } \]
\[ \xymatrix{ 1 \ar[d]_{\langle \false, \true \rangle} \ar[rd]^{\false} \\
              \bool \times \bool \ar[r]_-{\andb} & \bool
            }
\qquad \xymatrix{ 1 \ar[d]_{\langle \false, \true \rangle} \ar[rd]^{\true} \\
              \bool \times \bool \ar[r]_-{\orb} & \bool
            } \]
\[ \xymatrix{ 1 \ar[d]_{\langle \false, \false \rangle} \ar[rd]^{\false} \\
              \bool \times \bool \ar[r]_-{\andb} & \bool
            }
\qquad \xymatrix{ 1 \ar[d]_{\langle \false, \false \rangle} \ar[rd]^{\false} \\
              \bool \times \bool \ar[r]_-{\orb} & \bool
            } \]

\item Мы видели, что объекты $2$ и $1$ могут быть изоморфны. Если $2$ является булевским объектом, то это все равно может произойти, но эту ситуацию легко отследить.

Пусть $\C$ -- категория с конечными произведениями.
Докажите, что следующие утверждения эквивалентны:
\begin{enumerate}
\item $\C$ -- категория предпорядка.
\item В $\C$ терминальный объект является булевским.
\item В $\C$ существует булевский объект, такой что $\true = \false$.
\end{enumerate}

\textit{Решение}
\begin{itemize}
\item[$a\Rightarrow b)$] Так как в категории предпорядка морфизм между двумя объектами единственный, то можно взять $Bool = 1, true = false = id$. Тогда $\forall f, g : A \to B ~~\exists h = f \circ \pi_2$
\item[$b\Rightarrow c)$] В терминальный объект существует только одна стрелка
\item[$c\Rightarrow a)$] В определении Bool $\langle true \circ !, id \rangle = \langle false \circ !, id \rangle$. Значит для любых 
\end{itemize}

\item Докажите, что в любой декартово замкнутой категории $\C$ выполнены следующие утверждения:
\begin{enumerate}
\item Для любого объекта $A$ существует изоморфизм $A^1 \simeq A$.\\
Пусть есть стрелка $f: A \times 1 \to B$. Тогда всегда существует уникальная стрелка $A \times 1 \to B \times 1$ --- $\langle f, id\rangle$ такая, что $\pi_1 \circ \langle f, id\rangle = f$. Значит пара $(B, \pi_1)$ является эеспонентой $B^1$ (по определению экспоненты). То есть $B^1 \simeq B$
\item Для любых объектов $A$, $B$ и $C$ существует изоморфизм $A^{B \times C} \simeq (A^B)^C$.

Определим $ev: (A^B)^C\times (B \times C) \to A$ следующим образом:\\
$ (A^B)^C\times (B \times C) \to_{iso} ((A^B)^C\times C) \times B \to_{\langle ev_1 , id\rangle}
A^B \times B \to_{ev_2} A$\\
Если $f: \Gamma \times (B \times C) \to A$, то $\exists! g: \Gamma\times C \to A^B$, который коммутирует с $ev_2$. Значит $\exists! h: \Gamma \to (A^B)^C$, который коммутирует с $ev_1$. Из построению $ev$ следует, что $h$ --- единственный, который коммутирует с ev.\\
По универсальному свойству $A^{B\times C} \simeq (A^B)^C$

\item Умножение дистрибутивно над сложением, то есть для любых объектов $A$, $B$ и $C$ морфизм
\[ [\langle \pi_1, \inj_1 \circ \pi_2 \rangle, \langle \pi_1, \inj_2 \circ \pi_2 \rangle ] : (A \times B) \amalg (A \times C) \to A \times (B \amalg C) \]
является изоморфизмом, где $\inj_1 : B \to B \amalg C$ и $\inj_2 : C \to B \amalg C$ -- канонические морфизмы копроизведения, и
если $f : B \to X$, $g : C \to X$, то $[f,g] : B \amalg C \to X$ -- уникальный морфизм, удовлетворяющий $[f,g] \circ \inj_1 = f$ и $[f,g] \circ \inj_2$.\\

\item Если в $\C$ существует начальный объект 0, то для любого объекта $A$ существует изоморфизм $A^0 \simeq 1$.\\
$\forall X: Hom(X, A^0) \simeq Hom(X \times 0, A)\simeq Hom(0, A) \Rightarrow \\\Rightarrow \forall X |Hom(X, A^0)| = 1\Rightarrow A^0 = 1$
\item Если в $\C$ существует копроизведение $B \amalg C$, то для любого объекта $A$ существует изоморфизм $A^{B \amalg C} \simeq A^B \times A^C$.\\

Определим $ev: (A^B\times A^C)\times(B\amalg C) \to A$ следующим образом:\\
$(A^B\times A^C)\times(B\amalg C) \to_{iso} (A^B\times A^C \times B)\amalg (A^B\times A^C \times C) \to\\ \to_{[inj_1 \circ\pi_{1,3},inj_2 \circ \pi_{2,3}]} (A^B \times B)\amalg (A^C \times C) \to_{[ev_1, ev_2]} A \amalg A \to_{[id, id]} A$\\
Пусть $f : \Gamma \times (B\amalg C) \to A$. Тогда существует изоморфная пара морфизмов $g_1: Gamma\times B \to A$, $g_2:\Gamma \times C \to A$ (которые получаются композицией с $inj_1, inj_2$). Для $g_1, g_2$ существует единственная пара морфизвом $\Gamma \to A^B$ и $\Gamma \to A^C$ такая, что первый коммутирует с $ev_1$, а второй --- с $ev_2$ (точнее не сами морфизмы, а $\langle \cdot, id\rangle$). Тогда существует единственный морфизм $h: \Gamma \to (A^C\times A^B)$, что $\langle h, id\rangle$ коммутирует с  $ev$.\\
По универсальному свойству $A^{B\amalg C} \simeq A^B\times A^C$

\end{enumerate}


\item Докажите, что в декартово замкнутой категории объект $2$ всегда является булевским.

$2 \times A = (1 \amalg 1)\times A \simeq
(1 \times A)\amalg (1 \times A) \simeq A\amalg A$\\
Тогда в определении для Bool можно взять стрелку из определения копроизведения $f, g$. Она будет уникальной по универсальному свойству копроизведения.

\item Определите в произвольной декартово замкнутой категории комбинаторы $K$ и $S$, то есть следующие морфизмы:
\begin{align*}
K & : A \to A^B \\ 
S & : (C^B)^A \to (C^A)^{(B^A)}
\end{align*}
\begin{itemize}
\item
$ \exists \pi_1 : A \times B \to A \Rightarrow \exists! K : A \to A^B $
\item
Есть стрелка $ev_1: B^A\times A \to B$\\
Есть стрелка $ev_2: (C^B)^A \times A \to C^B$\\
Есть стрелка $ev_3: C^B \times B \to C$\\
Значит есть стрелка $(C^B)^A \times B^A \times A \to C$\\
Ее можно построить так: $(C^B)^A \times B^A \times A \to (C^B)^A \times B^A \times A\times A \to\\
\to ((C^B)^A\times A) \times (B^A \times A) \to C^B \times B \to C$\\
(так как категория декартова, все переходу существуют)\\
Тогда существует уникальная стрелка $(C^B)^A \times B^A \to C^A$\\
Тогда существует уникальная стрелка $(C^B)^A \to (C^A)^{(B^A )}$
\end{itemize}

\item Одна из аксиом арифметики Пеано говорит, что функция $\suc$ должна быть инъективной.
Докажите, что в любой декартово замкнутой категории с объектом натуральных чисел морфизм $\suc$ является расщепленным мономорфизмом.

\item Одна из аксиом арифметики Пеано говорит, что ни для какого $x$ не верно, что $\zero = \suc(x)$.
В произвольной декартово замкнутой категории это может быть верно, но только если она является категорией предпорядка.
Докажите, что следующие утверждения эквивалентны.
\begin{enumerate}
\item $\C$ -- категория предпорядка.\\

\item В $\C$ терминальный объект является объектом натуральных чисел.
\item В $\C$ существует объект натуральных чисел, такой что для любого $x : 1 \to \mathbb{N}$ верно, что $\zero = \suc \circ x$.
\item В $\C$ существует объект натуральных чисел, такой что для некоторого $x : 1 \to \mathbb{N}$ верно, что $\zero = \suc \circ x$.
\end{enumerate}

\item Докажите, что если в декартово замкнутой категории существует все малые копроизведения, то в ней существует объект натуральных чисел.

\item Определите в произвольной декартово замкнутой категории с объектом натуральных чисел морфизм сложения $+ : \mathbb{N} \times \mathbb{N} \to \mathbb{N}$, удовлетворяющий следующим условиям:
\[ \xymatrix{ \mathbb{N} \ar[d]_{\langle \zero \circ !_\mathbb{N}, id_\mathbb{N} \rangle} \ar[rd]^{id_\mathbb{N}} \\
              \mathbb{N} \times \mathbb{N} \ar[r]_-{+} & \mathbb{N}
            }
\qquad
   \xymatrix{ \mathbb{N} \times \mathbb{N} \ar[r]^-{+} \ar[d]_{\suc \times id_\mathbb{N}} & \mathbb{N} \ar[d]^{\suc} \\
              \mathbb{N} \times \mathbb{N} \ar[r]_-{+} & \mathbb{N}
            } \]
Докажите, что сложение коммутативно и ассоциативно, то есть, что коммутируют следующие диаграммы:
\[ \xymatrix{ \mathbb{N} \times \mathbb{N} \ar[r]^{\langle \pi_2, \pi_1 \rangle} \ar[rd]_{+} & \mathbb{N} \times \mathbb{N} \ar[d]^{+} \\
                                                                           & \mathbb{N} \times \mathbb{N}
            } \]
\[ \xymatrix{ (\mathbb{N} \times \mathbb{N}) \times \mathbb{N} \ar[r]^{\simeq} \ar[d]_{+ \times id_\mathbb{N}} & \mathbb{N} \times (\mathbb{N} \times \mathbb{N}) \ar[r]^-{id_\mathbb{N} \times +} & \mathbb{N} \times \mathbb{N} \ar[d]^{+} \\
              \mathbb{N} \times \mathbb{N} \ar[rr]_{+} & & \mathbb{N}
            } \]

\end{enumerate}

\end{document}
