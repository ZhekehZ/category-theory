\documentclass[draft]{article}
\usepackage[russian]{babel}
\usepackage[utf8]{inputenc}
\usepackage{cmap}
\usepackage{amsfonts}
\usepackage{amssymb}
\usepackage{amsmath}
\usepackage[all]{xy}

\newcommand{\cat}[1]{\mathbf{#1}}
\renewcommand{\C}{\cat{C}}
\newcommand{\D}{\cat{D}}
\newcommand{\y}{\cat{y}}
\newcommand{\Set}{\cat{Set}}
\newcommand{\Grp}{\cat{Grp}}
\newcommand{\Ab}{\cat{Ab}}
\newcommand{\Mat}{\cat{Mat}}
\newcommand{\fs}[1]{\mathrm{#1}}
\newcommand{\op}{\fs{op}}
\newcommand{\Num}{\cat{Num}}

\newcommand{\Conus}{\mathrm{Conus}}
\newcommand{\Hom}{\mathrm{Hom}}
\newcommand{\limit}{\mathrm{lim}}
\newcommand{\colim}{\mathrm{colim}}

% \tolerance=500
% \setlength{\emergencystretch}{3em}

\newenvironment{tolerant}[1]{\par\tolerance=#1\relax}{\par}

\begin{document}

\title{Задания}
\maketitle

\begin{enumerate}

\item Докажите, что вложение Йонеды сохраняет пределы.

\item Докажите, что вложение Йонеды сохраняет экспоненты. То есть, если $a$, $b$ -- объекты $\C$ такие, что $b^a$ существует,
то $\y(b)^{\y(a)}$ тоже существует и определяется как $\y(b^a)$.

\item Докажите, что коллекция объектов вида $\y a$ является генератором для категории предпучков.

\[\xymatrix{ \y a \ar[r]^{s} & F \ar@<+0.5ex>[r]^{f} \ar@<-0.5ex>[r]_{g}  & G } \]
$Hom(\y a, F) \simeq F_a,~~Hom(\y a, G) \simeq G_a$.\\
$f \simeq f': F_a \to G_a, ~~g \simeq g': F_a \to G_a$\\
$\forall s' \in F_a: f'(s') = g'(s') \Rightarrow f' = g' \Rightarrow f = g$

\item Определите категорию $\C$, такую что $\Set^{\C^\op}$ эквивалентна категори рефлексивных графов.

\[
\xymatrix{
	V 
	\ar@(ld,lu)^{id}
	\ar@/^1.5pc/[rr]^{src}
	\ar@/_1.5pc/[rr]_{dst}
	&& 
	E 
	\ar@(rd,ru)_{id}	
	\ar@(ul,ur)^{l}	
	\ar@(dl,dr)_{r}	
	\ar[ll]^{i}
}
\]

$G_E$ -- ребра, $G_V$ -- вершины.\\
$dst, src: ~G_E \to G_V$ \\
$i: ~G_V \to G_E$\\
$dst \circ i = id$\\
$src \circ i = id$\\
$i \circ dst = r$\\
$i \circ src = l$\\
$src \circ l = src$\\
$dst \circ l = src$\\
$dst \circ r = dst$\\
$src \circ r = dst$\\
$l \circ i = i$\\
$r \circ i = i$


\item Докажите, что функтор $F : \Set^{\C^\op} \to \D$ является левым сопряженным тогда и только тогда, когда он сохраняет копределы.

Достаточно доказать, что если он сохраняет копределы, то он левый сопряженный.\\

Пусть $X \in \Set^{\C^{op}}$. Тогда по ко-лемме $X = colim_a~\y a$.\\
$\Hom(F(X), Y) = \Hom(F(colim_a~\y~a), Y) =\\
= \Hom(colim_a~F(\y~a), Y) = lim_a\Hom(F(\y~a), Y)=\\
= lim_a\Hom(\y~a, Hom(F(\y~\_), Y)) = \\= \Hom(colim_a~\y~a, Hom(F(\y~\_), Y)) = \Hom(X, Hom(F(\y~\_), Y)) =\\=: \Hom(X, U(Y))$\\

$U = \Hom(F(\y~\_), \_)~~:~~D\to \Set^{C^{op}}$ --- правый сопряженный

\item Докажите, что функтор $\Set^{\C^\op}$ является свободным копополнением $\C$, то есть, что для любой кополной категории $\D$ и любого функтора $F : \C \to \D$
существует уникальный (с точностью до изоморфизма) функтор $\widetilde{F} : \Set^{\C^\op} \to \D$, сохраняющий копределы, и такой, что следующая диаграмма коммутирует (с точностью до изоморфизма функторов):
\[ \xymatrix{ \C \ar[r]^F \ar[d]_{\cat{y}}                 & \D \\
              \Set^{\C^\op} \ar@{-->}[ur]_{\widetilde{F}}
            } \]


Пусть такой $\widetilde{F}$ существует и пусть $X \in \Set^{\C^{op}} = colim_a ~\y~a$. Тогда\\
$\widetilde{F}(X) = \widetilde{F}(colim_a~\y~a) = colim_a~(\widetilde{F}(\y~a)) = colim_a~F_a$.

Тогда можно взять $\widetilde{F}(X) = colim_a~F_a$ как определение $\widetilde{F}$. Оно корректно, так как в $\D$ существуют копределы.

\end{enumerate}

\end{document}
