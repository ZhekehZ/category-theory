\documentclass[draft]{article}
\usepackage[russian]{babel}
\usepackage[utf8]{inputenc}
\usepackage{cmap}
\usepackage{amsfonts}
\usepackage{amssymb}
\usepackage{amsmath}
\usepackage[all]{xy}

\newcommand{\cat}[1]{\mathbf{#1}}
\renewcommand{\C}{\cat{C}}
\newcommand{\D}{\cat{D}}
\newcommand{\Set}{\cat{Set}}
\newcommand{\FinSet}{\cat{FinSet}}
\newcommand{\Mon}{\cat{Mon}}
\newcommand{\Grp}{\cat{Grp}}
\newcommand{\Ab}{\cat{Ab}}
\newcommand{\Mat}{\cat{Mat}}
\newcommand{\Num}{\cat{Num}}

\begin{document}

\title{Задания}
\maketitle

\begin{enumerate}

\item Пусть $F : \C \to \D$ -- некоторый функтор.
Какие из следующих утверждений верны?
Как изменится ответ, если предположить, что $F$ -- эквивалентность категорий?
\begin{enumerate}
\item Если $f : X \to Y$ -- мономорфизм в $\C$. то $F(f)$ -- мономорфизм в $\D$.
\item Если $X$ -- (ко)предел диаграммы $D : \cat{J} \to \C$, то $F(X)$ -- (ко)предел диграммы $F \circ D : \cat{J} \to \D$.
\end{enumerate}

\item Пусть $\cat{Cat}$ -- категория малых категорий.
Ее объекты -- это малые категории.
Морфизмы в категории $\cat{Cat}$ -- это функторы между категориями.

Пусть $\cat{Graph}$ -- категория графов.
Ее объекты -- графы, то есть пары $(V,E)$, состоящие из множества вершин $V$ и функции $E$, сопоставляющей каждой паре вершин $x,y \in V$ множество $E(x,y)$ ребер из $x$ в $y$.

Морфизм графов $(V,E)$ и $(U,D)$ состоит из функции $f : V \to U$ и функции $f : E(x,y) \to D(f(x), f(y))$ для всех $x,y \in V$.
Композиция и тождественные морфизмы определены очевидным образом.

Определите забывающий функтор из $\cat{Cat}$ в $\cat{Graph}$.
Докажите, что этот функтор строгий.

\item В лекции определялся функтор $I : \Mon \to \Grp$ обратимых элементов моноида.
\begin{enumerate}
\item Является ли $I$ строгим? Докажите это.\\
Рассмотрим два моноида: первый $M_1$ --- моноид из строк над конечным алфавитом с операцией конкатенации; второй $M_2$ --- $(\mathbb{Z}, +)$.\\

$I(M_1)$ --- тривиальный моноид; $I(M_2) = M_2$.

Гомоморфизмы $f_1(s) = length(s),~~f_2(s) = 2\cdot length(s)$ отобразятся в один и тот же (единственный) гомоморфизм $f(x) = 0$.

$I$ не строгий.

\item Является ли $I$ полным? Докажите это.
\end{enumerate}

Рассмотрим два моноида: \\
$M_1$ --- моноид, состоящий из строк над алфавитом $\lbrace a,b,c,x \rbrace$ с операцией композиции, которая конкатенирует строки и удаляет все пары одинаковых подряд идущих символов, кроме $xx$.
$M_1$ ---  моноид, состоящий из строк над алфавитом $\lbrace a,b,c \rbrace$ с теми же свойствами (уже без $xx$) и развернутой операцией композиции.

Тогда $I(M_1) = I(M_2)$.\\
Возьмем функцию $f(a) = a^-1$. Найдем ее прообраз $g$.\\

$ax \neq 1 \Rightarrow g(ax) = g(x) * a \neq $



Если $a, b \in I(M_1)$, то $a*b \in I(M_1) \Rightarrow f'(a*b) = f(a*b) = f(a)*f(b)= f'(a)*f'(b)$\\
Если $a \notin I(M_1), b \in I(M_1)$, то $f'(a*b) = f'(a) * f(b) = f'(a) * f'(b)$\\
Если $a \notin I(M_1), b \notin I(M_1)$ и $b = c*d, c\notin I(M_1), d \in I(M_1)$, то $f'(a*b) = f'(a*c*d) = f'(a*c) * f(d)= f(d)$


\item Докажите, что если $F : \C \to \C$ -- некоторый эндофунктор, то начальная $F$-алгебра $X$ удовлетворяет уравнению $X \simeq F(X)$.

Пусть $(X_0, \alpha)$ --- начальный объект. Тогда рассмотрим алгебру \\$(F(X_0), F(\alpha))$. Тогда существует кникальный $f$, для которого диаграмма ниже коммутирует.\\
\[
\xymatrix{
F(X_0) \ar[r]^{\alpha} \ar[d]_{F(f)} & X_0 \ar[d]_{f} \\
F(F(X_0)) \ar[r]_{F(\alpha)} & F(X_0)
}
\]
$ \alpha \circ f:  X_0 \to X_0 $ --- морфизм в категории F-алгебр. Так как $(X_0, \alpha)$ --- начальный, то $\alpha \circ f = id$.\\

Тогда из диаграммы получаем:\\
$F(\alpha) \circ F(f)= F(\alpha \circ f) = F(id) = id = f \circ \alpha$

То есть $\alpha$ --- изо, а значит $X_0\simeq F(X_0)$


\end{enumerate}

\end{document}
