\documentclass[draft]{article}
\usepackage[russian]{babel}
\usepackage[utf8]{inputenc}
\usepackage{cmap}
\usepackage{amsmath}
\usepackage{amsfonts}
\usepackage[all]{xy}

\newcommand{\cat}[1]{\mathbf{#1}}
\renewcommand{\C}{\cat{C}}
\newcommand{\Set}{\cat{Set}}
\newcommand{\FinSet}{\cat{FinSet}}
\newcommand{\Grp}{\cat{Grp}}
\renewcommand{\Vec}{\cat{Vec}}
\newcommand{\Mat}{\cat{Mat}}
\newcommand{\Num}{\cat{Num}}
\newcommand{\fs}[1]{\mathrm{#1}}
\newcommand{\Hom}{\fs{Hom}}
\newcommand{\id}{\fs{id}}

\newenvironment{tolerant}[1]{\par\tolerance=#1\relax}{\par}

\begin{document}

\title{Задания}
\maketitle

Если $M$ -- моноид, то мы будем обозначать $\C_M$ категорию с одним объектом $*$ и множеством морфизмов $\Hom_{\C_M}(*, *) = M$, операция композиции и тождественный морфизм в которой определяются как соответствующие операции в $M$.

Предпорядок $(X,\leq)$ -- это множество $X$ с рефлексивным и транзитивным бинарным отношением $\leq$.
Задать структуру предпорядка на множестве -- это то же самое, что и задать на нем структуру категории, в которой между любой парой объектов существует максимум один морфизм.
Если $(X,\leq)$ -- предпорядок, то мы будем обозначать соответствующую ему категорию как $\C_{(X,\leq)}$.
Множество объектов этой ктаегории равно $X$, а множество морфизмов $\Hom_{\C_{(X,\leq)}}(x, y)$ состоит из одного элемента, если $x \leq y$, и пусто в противном случае.

\begin{enumerate}

\item Изоморфны ли следующие объекты категории $\Lambda_\fs{ID}$? Если да, напишите функции, устанавливающие изоморфизм.
\begin{enumerate}
\item $\fs{Bool}$ и $\fs{Maybe\ Bool}$.
\item $\fs{Either\ Bool\ Bool}$ и $\fs{Bool} \times \fs{Bool}$.
\item $\fs{Nat}$ и $\fs{Maybe\ Nat}$.
\item $\fs{Nat}$ и $\fs{List\ Nat}$.
\end{enumerate}

\textit{\textbf{Решение}}
\begin{enumerate}
\item Нет
\item Нет
\item Нет
\item Нет
\end{enumerate}


\item Пусть $M$ -- некоторый моноид.
Определим тогда категорию $\C_M$ как категорию с одним объектом и множеством морфизмов равным $M$.
Композиции и тождественный морфизм определяются из структуры моноида.
Какие морфизмы являются изоморфизмами в следующих категориях?
\begin{enumerate}
\item $\C_{(\mathbb{N},+)}$.
\item $\C_{(\mathbb{N},*)}$.
\item $\C_{(\mathbb{Z},+)}$.
\item $\C_{(\mathbb{Z},*)}$.
\item $\C_{(\mathbb{Q},+)}$.
\item $\C_{(\mathbb{Q},*)}$.
\end{enumerate}

\item Предпорядок называется частичным порядком, если из условия, что $x \leq y$ и $y \leq x$, следует, что $x = y$.
Чему в категориальных терминах соотвествует это свойство?
(Другими словами, утверждается, что предпорядок $(X,\leq)$ является порядком тогда и только тогда, когда категория $\C_{(X,\leq)}$ обладает некоторым свойством,
которое обсуждалось на лекции. Что это за свойство?)

\item Опишите следующие моноиды и группы:
\begin{enumerate}
\item $\fs{Aut}_\Set(A)$, где $A$ -- множество букв русского алфавита.
\item $\fs{Aut}_{\FinSet}(A)$, где $A$ -- множество букв русского алфавита.
\item $\fs{Endo}_{\C_M}(*)$, где $M$ -- некоторый моноид.
\item $\fs{Endo}_\Grp(\mathbb{Z})$.
\item $\fs{Aut}_\Grp(\mathbb{Z})$.
\item $\fs{Endo}_\cat{Ring}(\mathbb{Z})$, где $\cat{Ring}$ -- категория колец с единицей. 

$\forall f \in \fs{Endo}_\cat{Ring}(\mathbb{Z}):$\\
$\forall z\in\mathbb{Z}. f(z) = z\cdot f(1)$\\
$\forall z\in\mathbb{Z}. z\cdot f(1) = f(z) = f(1 \cdot z) = f(1)\cdot z\cdot f(1) \Rightarrow f(1) = 0 \lor f(1) = 1$\\
\textit{Значит} $~~f(.) \equiv 0 \lor f = id$

\item $\fs{Aut}_\C(X)$, где $\C$ -- скелетная категория, и $X$ -- произвольный объект $\C$.

Просто группа (??)

\item $\fs{Endo}_\Vec(\mathbb{R}^n)$.

$\forall f \in \fs{Endo}_\Vec(\mathbb{R}^n):$\\
$f : \mathbb{R}^n \to \mathbb{R}^n$ --- линейный оператор\\
$\fs{Endo}_\Vec(\mathbb{R}^n)$ \textit{ изоморфен моноиду из матриц $n \times n$ с операцией умножения}

\item $\fs{Aut}_\Num(n)$.

$\forall f \in \fs{Aut}_\Num(n):$
$f \in [0 .. n]^n$ --- изоморфизм\\
\textit{Значит $\fs{Aut}_\Num(n)$ --- множество перестановок из n элементов}

\item $\fs{Endo}_{\C_{(X,\leq)}}(x)$, где $x$ -- произвольный элемент $X$.

$\fs{Endo}_{\C_{(X,\leq)}}(x) = \lbrace x \leq x \rbrace$ операцией композиции по транзитивности

\end{enumerate}

\item Какие из следующих категорий являются скелетными: $\Set$, $\FinSet$, $\Grp$, $\Vec$, $\Lambda$, $\Mat$, $\Num$?

\begin{enumerate}
\item $\Set$ --- нет, так как $\lbrace 0 \rbrace$ изоморфен $\lbrace 1 \rbrace$, но не равны
\item $\FinSet$ --- нет, аналогично $\Set$
\item $\Grp$ --- нет, $(\mathbb{R}, +)$ изоморфна $(\mathbb{R}^{> 0}, \cdot)$
\item $\Vec$ --- нет, $i\mathbb{R}$ изоморфен $\mathbb{R}$, но не равен
\item $\Lambda$ --- нет, $a \to a \to b \to a$ изоморфен $a \to b \to a \to a$:\\
$f = \lambda~h~a~b.~h~b~a$\\
$f \circ f = \lambda~x.~f(f x) = \lambda~x.~f(\lambda~b~a.~x~a~b)= \lambda~x~a~b.~x~a~b = \lambda~x.~x = id$
\item $\Mat$ --- да
\item $\Num$ --- да
\end{enumerate}

\item Какие из следующих категорий являются группоидами: $\Set$, $\FinSet$, $\Grp$, $\Vec$, $\Lambda$, $\Mat$, $\Num$?

\begin{enumerate}
\item $\Set$ --- нет, есть инъекции, например
\item $\FinSet$ --- нет, есть инъекции, например
\item $\Grp$ --- нет, можно построить гомоморфизм в группу из одного нейтрального элемента
\item $\Vec$ --- нет, $\mathbb{R}^3 \to \mathbb{R}^2$
\item $\Lambda$ --- нет, есть терм с типом $(a\to a\to b)\to (a\to b\to b)$
\item $\Mat$ --- нет, есть необратимые матрицы
\item $\Num$ --- нет, $(2) \in \Hom_\Num(1, 2)$ не является изоморфизмом, так как $\Hom_\Num(2, 1) = \lbrace (1, 1) \rbrace$
\end{enumerate}


\item Какие из следующих категорий могут быть скелетными и в каких случаях?
\begin{enumerate}
\item Дискретные категории.\\
		Всегда скелетные
\item Категории вида $\C_M$.\\
		Всегда скелетные
\item Категории предпорядка.\\
		Если это частичный порядок
\item Группоиды.\\
		Когда $ \Hom(A, B) \neq \emptyset \Rightarrow A = B$
\end{enumerate}


\item Какие из следующих категорий могут быть группоидами и в каких случаях?
\begin{enumerate}
\item Дискретные категории.\\
		Всегда группоиды.
\item Категории вида $\C_M$.\\
		Когда $M$ --- группа
\item Категории предпорядка.\\
		Когда это дискретные категории
\item Скелетные категории.\\
		Когда это дискретные категории
\end{enumerate}

\item Пусть $f, f' : X \to Y$ и $g, g' : Y \to X$ -- морфизмы в некоторой категории $\C$.
Докажите, что если диаграммы
\[ \xymatrix{ & Y \ar[dr]^g & \\
              X \ar[ur]^f \ar[rr]_{\id_X} & & X
            }
\qquad
\xymatrix{ & X \ar[dr]^{f'} & \\
           Y \ar[ur]^{g'} \ar[rr]_{\id_Y} & & Y
            } \]
коммутируют и $f = f'$, то $X$ и $Y$ изоморфны.

\textit{Перевернем первый треугольник и склеим со вторым. }\\
\textit{Так как они коммутировали, то полученная диаграмма будет коммутировать.}\\
\textit{Отсюда: }  
$id_X \circ g' = g \circ id_Y \Rightarrow g = g'$\\
$g \circ f = id_X,~f\circ g = id_Y \Rightarrow X$ \textit{ изоморфен } $Y$

\item Приведите пример, показывающий, что условие $f = f'$ в предыдущем задании является необходимым.

\item Какие из следующих категорий являются малыми: $\Set$, $\FinSet$, $\Grp$, $\Vec$, $\Lambda$, $\Mat$, $\Num$, $\C_M$, $\C_{(X,\leq)}$?

Все, кроме: $\Set, \FinSet, \Grp, \Vec$

\item
\tolerant{500}{
Какие из следующих категорий являются локально малыми: $\Set$, $\FinSet$, $\Grp$, $\Vec$, $\Lambda$, $\Mat$, $\Num$, $\C_M$, $\C_{(X,\leq)}$?
}

Все

\end{enumerate}

\end{document}
