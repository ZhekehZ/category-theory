\documentclass[draft]{article}
\usepackage[russian]{babel}
\usepackage[utf8]{inputenc}
\usepackage{cmap}
\usepackage{amsfonts}
\usepackage[all]{xy}
\usepackage{ bbold }

\newcommand{\cat}[1]{\mathbf{#1}}
\renewcommand{\C}{\cat{C}}
\newcommand{\Set}{\cat{Set}}
\newcommand{\Grp}{\cat{Grp}}
\newcommand{\Ab}{\cat{Ab}}
\newcommand{\Vec}{\cat{Vec}}
\newcommand{\Hask}{\cat{Hask}}
\newcommand{\Mat}{\cat{Mat}}
\newcommand{\Num}{\cat{Num}}

\begin{document}

\title{Задания}
\maketitle

\begin{enumerate}

\item Опишите в категории (пред)порядка следующие конструкции:
\begin{enumerate}
\item Терминальные объекты.\\
\textit{Наибольший элемент}
\item Произведения объектов.\\
\textit{Точная нижняя граница}
\end{enumerate}

\item Пусть в категории $\C$ существует терминальный объект 1.
Докажите, что для любого объекта $A$ в $\C$ существует произведение $A \times 1$.

$A \times 1 = A, \pi_1 = id, \pi_2 = (\textit{единственный морфизм } A \to 1)$\\
$\forall C, f_1 : C \to A, f_2: C \to 1$\\ 
$\exists h = f_1$. Одна половина коммутирует, так как $id \circ f_1 = f1$, вторая — так как там терминальный объект. h единственный, так как $id \circ h = f_1 \Rightarrow h = f_1$

\item Докажите, что любой морфизм из терминального объекта является мономорфизмом.

В терминальный объект существует только одна стрелка из другого объекта.

\item Пусть в категории $\C$ существует терминальный объект 1 и некоторый морфизм $1 \to B$.
Докажите, что любая проекция $\pi_1 : A \times B \to A$ является эпиморфизмом.

\[ \xymatrix{
& A=A\times 1 \ar[dl]\ar[dr]^{id}\ar[d] & & \\
B & A\times B \ar[l]\ar[r]_{\pi_1} & A \ar@<0.5ex>[r]^f\ar@<-.5ex>[r]_g & C
            } \]

\item Докажите, что в $\Ab$ существуют все произведения.

$(A_1, *) \times (A_2, +) = (  A_1\times  A_2 , \langle*, +\rangle  )$\\
проекции тривиальные.
свойство выполняется для функции $h(x) = (f(x), g(x))$. Она единственная, так как любую функцию, действующую в множество пар можно разбить на 2 функции и тогда $id \circ f = f, id \circ g = g$

\item Докажите, что два определения уравнителей, приводившихся в лекции, эквивалентны.

\begin{itemize}
\item[$\Rightarrow$]$e$ --- моно $ \Rightarrow  \exists!ke$\\ если e — моно, то k — единственный, для которого диаграмма коммутирует ($e \circ k = h = e \circ k'\Rightarrow k = k'$)
\item[$\Leftarrow$)] $\exists!k \Rightarrow e$ --- моно\\
Пусть $e \circ w = e \circ t$ для неких $t, w$. Тогда $e \circ w$ можно подставить вместо h в определение уравнителя. Тогда $\exists! k: e\circ k = e \circ w$. Получается, что $k = w = t$. Значит e --- моно.
\end{itemize}

\item Докажите, что уравнитель пары стрелок $f,g : A \to B$ уникален с точностью до изоморфизма.
То есть, если $e_1 : E_1 \to A$ и $e_2 : E_2 \to A$ -- два уравнителя $f$ и $g$, то существует уникальный изоморфизм $i : E_1 \to E_2$ такой, что $e_2 \circ i = e_1$.

По определению уравнителя: $\exists! ~k_1.~e_1\circ k_1 = e_2, ~~\exists! ~k_2.~e_2\circ k_2 = e_1$.\\
Так как уравнители --- мономорфизмы, то:\\
$e_1\circ k_1 \circ k_2 = e_1 ~~\Rightarrow~~ k_1 \circ k_2 = id$\\
$e_2\circ k_2 \circ k_1 = e_2 ~~\Rightarrow~~ k_2 \circ k_1 = id$\\
Значит $k_1, k_2$ --- изоморфизмы, причем единственные

\item Морфизм $h : B \to B$ называется \emph{идемпотентным}, если $h \circ h = h$.
Докажите следующие факты:
\begin{enumerate}
\item Если $f : A \to B$ и $g : B \to A$ -- такие, что $g \circ f = id_A$, то $h = f \circ g$ является идемпотентным. 

$h \circ h = (f \circ g)\circ (f\circ g) = f \circ (g\circ f)\circ g = f \circ g = h$
\item Если в категории есть уравнители, то обратное верно.
Конкретно, для любого идемпотентного морфизма $h : B \to B$ существуют $f : A \to B$ и $g : B \to A$ такие, что $g \circ f = id_A$ и $f \circ g = h$.\\

Построим уравнитель для h, id. Так как $h \circ h = id \circ h$, то $\exists f, \exists!g$
\[ \xymatrix{ B \ar[dr]^{h}\ar[d]_g & & \\
              A \ar@{^{(}->}[r]_f & B \ar@<-.5ex>[r]_{id}\ar@<.5ex>[r]^{h} & B
            } \]

$f \circ g = h$\\
$f \circ g \circ f = h \circ f$\\
Но $h \circ f = id \circ f$ м f --- моно, а значит\\
$f \circ g \circ f = f$\\
$g \circ f = id$

\end{enumerate}

\item Докажите, что любой расщепленный мономорфизм регулярен.

Пусть  $f$ --- регулярный моно и $g\circ f = id$. Рассмотрим диаграмму

\[ \xymatrix{ C \ar[dr]^{h}\ar@{-->}[d]_{g\circ h} & & \\
              A \ar@{^{(}->}@<.5ex>[r]^f & B \ar@<.5ex>[l]^g \ar@<-.5ex>[r]_{id}\ar@<.5ex>[r]^{f\circ g} & B
            } \]
Достаточно проверить, что эта диаграмма коммутирует для любого h (точнее даже нужно проверить только треугольник).\\
$f\circ (g\circ h) = (f \circ g) \circ h = h \circ id = h$

\item Мономорфизм $f : A \to B$ называется \emph{сильным}, если для любой коммутативного квадрата, где $e : C \to D$ является эпиморфизмом,
\[ \xymatrix{ C \ar[r] \ar[d]_e      & A \ar[d]^f \\
              D \ar[r] \ar@{-->}[ur] & B
            } \]
существует стрелка $D \to A$ такая, что диаграмма выше коммутирует.

Докажите, что любой регулярный мономорфизм силен.

\[ \xymatrix{ 
C \ar@{->>}[r]^e \ar[dr]_t & D \ar@{~>}[dr]^w \ar@{-->}[d]_k &   & \\
  & A \ar@{^{(}->}[r]_f & B \ar@<-0.5ex>[r]_g \ar@<0.5ex>[r]^h & K
 } \]
Чтобы доказать существование $k$, достаточно показать, что $h\circ w = g \circ w$\\
$h\circ f \circ t = g \circ f \circ t~~~$ --- тк f --- уравнитель\\
$h\circ w \circ e = g \circ w \circ e~~~$ --- тк квадрат коммутативен\\
$h\circ w = g \circ w~~~$ --- тк е --- эпи

\item Мономорфизм $f : A \to B$ называется \emph{экстремальным}, если для любого эпиморфизма $e : A \to C$ и любого морфизма $g : C \to B$ таких, что $g \circ e = f$, верно, что $e$ -- изоморфизм.

Докажите, что любой сильный мономорфизм экстремален.

Так как f --- сильный, то для любых e, g, для которых квадрат коммутативен, надется соответствующий h
\[ \xymatrix{ A \ar[r]^{id} \ar[d]_e      & A \ar[d]^f \\
              D \ar[r]_g \ar@{-->}[ur]^h & B
            } \]
$h \circ e = id$\\
$e \circ h \circ e = e$\\
$e \circ h = id~~~$ --- так как е --- эпи\\
Получается, что $e, h$ --- изо

\item Докажите, что если в категории все мономорфизмы регулярны, то она сбалансирована. Можно ли усилить это утверждение?

Если f --- регулярный моно- эпиморфизм, то он является уравнителем для неких g, h. Причем, так как f --- эпи, то g = h. Тогда по определению уравнителя $\exists!k: f \circ k = id$, то есть f --- расщепленный моно- эпи-, а значит изо.

\item Докажите, что в $\Set$ все мономорфизмы регулярны.

Пусть $f : A \to B$ --- моно. Если f --- эпи, то f --- изо, а значит расщепленный, а значит регулярен (пункт 9). \\
Если f --- не эпи, то f --- уравнитель для $\mathbb{1}_B$ и $\mathbb{1}_{f(A)}$

\item Докажите, что в $\Ab$ все мономорфизмы регулярны.

Пусть $f : A \to B$ --- моно. f --- уравнитель для $\mathbb{0}$ и некоторого $g$, где $g: B \to C, ~ker(g) = f(A)$\\
Так как $B$ --- абелева, значит $f(A)$ --- нормальная, значит можно определить $g$ так:\\
$g (b) = b\cdot f(A) ~:~ B \to B/f(A)$ 

\end{enumerate}

Бонусные задания:

\begin{enumerate}

\item Докажите, что если в категории $\C_M$ существуют бинарные произведения и моноид $M$ нетривиален, то он бесконечен.

%Пусть $1, a \in M, ~1\neq a$.\\
%Существует произведение:
%\[ \xymatrix{ 
%& \ast \ar[dr]^{g} \ar[dl]_{f} \ar@{-->}^{\exists! k}[d]     & \\
%\ast & \ast \ar[r]_{\pi_2} \ar[l]^{\pi_1} & \ast
%} \]
%$\pi_1 \neq \pi_2:$\\
%Пусть $\pi_1 = \pi_2 = \pi$. Тогда $\exists! k: ~\pi\circ k = 1, ~\pi\circ k = a ~~\Rightarrow~~ a = 1,$ противоречие
%
%Пусть $j: j^2 = 1 \land j \neq 1$. Тогда рассмотрим произведение при  $f = j\circ \pi_2, g = \pi_1$.\\
%$\exists! \kappa~~\pi_1 \circ \kappa = j\circ \pi_2~~\land~~\pi_2 \circ \kappa = \pi_1$.\\
%Тогда $\pi_1 \circ \kappa^4 = j\circ \pi_1 \circ \kappa^2 = j^2\circ \pi_1 = \pi_1$ \\
%и $\pi_2 \circ \kappa^4 = j \circ \pi_2 \circ \kappa^2 = j^2\circ \pi_2 = \pi_2$
%
%Так как $\exists!k = 1:~~\pi_i\circ k = \pi_i$, то $\kappa^4 = 1$.
%
%$\kappa \neq j:$\\
%Пусть $\kappa = j$. Тогда $\pi_1 = \pi_1 \circ j^2 = j \circ \pi_2 \circ j = j \circ \pi_1$\\
%$\pi_2 = \pi_2 \circ j^2 = \pi_1 \circ j = j \circ \pi_2$
%
%

\item Докажите, что если в категории $\C_M$ существуют бинарные произведения и моноид $M$ нетривиален, то для любого натурального $n > 1$ существует $x \in M$ такой, что $x \neq 1$ и $x^n = 1$.

\item Приведите пример нетривиального моноида $M$ такого, что в категории $\C_M$ существует бинарные произведения.

\end{enumerate}

\end{document}
