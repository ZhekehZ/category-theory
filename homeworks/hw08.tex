\documentclass[draft]{article}
\usepackage[russian]{babel}
\usepackage[utf8]{inputenc}
\usepackage{cmap}
\usepackage{amsfonts}
\usepackage{amssymb}
\usepackage{amsmath}
\usepackage[all]{xy}

\newcommand{\cat}[1]{\mathbf{#1}}
\renewcommand{\C}{\cat{C}}
\newcommand{\D}{\cat{D}}
\newcommand{\Set}{\cat{Set}}
\newcommand{\Grp}{\cat{Grp}}
\newcommand{\Ab}{\cat{Ab}}
\newcommand{\Mat}{\cat{Mat}}
\newcommand{\Num}{\cat{Num}}
\newcommand{\fs}[1]{\mathrm{#1}}
\newcommand{\Mod}[1]{#1\text{-}\cat{Mod}}

\newenvironment{tolerant}[1]{\par\tolerance=#1\relax}{\par}

\begin{document}

\title{Задания}
\maketitle

\begin{enumerate}

\item Пусть $T : \C \to \C$ -- монада.
Докажите, что функтор
\begin{align*}
& U^T : T\text{-}\cat{alg} \to \C \\
& U^T(A,h) = A
\end{align*}
является правым сопряженным к
\begin{align*}
& F^T : \C \to T\text{-}\cat{alg} \\
& F^T(A) = (T(A), \mu_A),
\end{align*}
и монада, соответствующая этому сопряжению, -- это просто $T$.

\item Докажите, что категория Клейсли $\cat{Kl}_T$ эквивалентна полной подкатегории $T\text{-}\cat{alg}$ на свободных $T$-алгебрах.



\item Опишите алгебраическую теорию, категория моделей которой эквивалентна категории рефлексивных графов, которые определялись в предыдущем ДЗ.

$T = (\mathcal{S},\mathcal{F}, \mathcal{A})$\\
$\mathcal{S} = \{v, e\}$\\
$\mathcal{F} = \{src~:~e\to v,~~dst ~:~ e \to v, ~~ id ~:~ v \to e\}$\\
$\mathcal{A} = \{ src ~(id~x) = x,~~ dst ~(id~x) = x\}$

\item Докажите, что для любой малой категории $\C$ категория функторов $\Set^{\C^\fs{op}}$ эквивалентна категории моделей некоторой алгебраической \tolerant{500}{ теории. }

\item Докажите, что категория $\Mod{\fs{Mon}}(\Mod{\fs{Mon}})$ моноидов в категории моноидов (в $\Set$) изоморфна категории коммутативных моноидов (в $\Set$).

\item Реализуйте на хаскелле монаду, аналогичную монаде абелевых групп на множествах.
Обратите внимание, что вам потребуется определить особым образом $\mathit{instance}\ \mathit{Eq}$ для типа монад.

\item Пусть $(A,*,1)$ -- моноид.
Тогда \emph{полумодуль} над моноидом $A$ -- это моноид $(M,+,0)$ вместе с операцией $\cdot : A \times M \to M$, удовлетворяющий следующим условиям:
\begin{itemize}
\item $r \cdot (x + y) = r \cdot x + r \cdot y$
\item $(r * s) \cdot x = r \cdot (s \cdot x)$
\item $1 \cdot x = x$
\end{itemize}
Забывающий функтор $\cat{Mod}_A \to \Set$ является правым сопряженным.
Следовательно существует монада полумодулей на категории $\Set$.
Реализуйте на хаскелле монаду, аналогичную этой монаде.

\item Пусть $(A,+,0,*,1)$ -- кольцо.
Тогда \emph{полумодуль} над кольцом $A$ -- это моноид $(M,+,0)$ вместе с операцией $\cdot : A \times M \to M$, удовлетворяющий следующим условиям:
\begin{itemize}
\item $r \cdot (x + y) = r \cdot x + r \cdot y$
\item $(r + s) \cdot x = r \cdot x + s \cdot x$
\item $0 \cdot x = 0$
\item $(r * s) \cdot x = r \cdot (s \cdot x)$
\item $1 \cdot x = x$
\end{itemize}
Забывающий функтор $\cat{Mod}_A \to \Set$ является правым сопряженным.
Следовательно существует монада полумодулей на категории $\Set$.
Реализуйте на хаскелле монаду, аналогичную этой монаде.

\item Реализуйте $\mathit{instance}\ \mathit{Monad}$ для типа $\mathit{Term}$:
\[ \mathit{data}\ \mathit{Term}\ a\ =\ \mathit{Var}\ a\ |\ \mathit{App}\ (\mathit{Term}\ a)\ (\mathit{Term}\ a)\ |\ \mathit{Lam}\ (\mathit{Term}\ (\mathit{Maybe}\ a)) \]
Реализуйте алгоритм нормализации для $\mathit{Term}$.

\end{enumerate}

\end{document}
