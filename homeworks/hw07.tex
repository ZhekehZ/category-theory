\documentclass[draft]{article}
\usepackage[russian]{babel}
\usepackage[utf8]{inputenc}
\usepackage{cmap}
\usepackage{amsfonts}
\usepackage{amssymb}
\usepackage{amsmath}
\usepackage[all]{xy}

\newcommand{\cat}[1]{\mathbf{#1}}
\renewcommand{\C}{\cat{C}}
\newcommand{\D}{\cat{D}}
\newcommand{\Set}{\cat{Set}}
\newcommand{\FinSet}{\cat{FinSet}}
\newcommand{\Grp}{\cat{Grp}}
\newcommand{\Mon}{\cat{Mon}}
\newcommand{\CMon}{\cat{CMon}}
\newcommand{\Ab}{\cat{Ab}}
\newcommand{\Mat}{\cat{Mat}}
\newcommand{\Num}{\cat{Num}}
\newcommand{\fs}[1]{\mathrm{#1}}
\newcommand{\id}{\fs{id}}

\newenvironment{tolerant}[1]{\par\tolerance=#1\relax}{\par}

\begin{document}

\title{Задания}
\maketitle

\begin{enumerate}

\item На второй лекции мы видели, что морфизм групп является мономорфизмом тогда и только тогда, когда мономорфизмом является соответствующая ему функция на множествах.
Сейчас мы можем обобщить это утверждение.
Забывающий функтор $U : \Grp \to \Set$ является правым сопряженным и строгим.
Для любого функтора, удовлетворяющего этим двум условиям, можно доказать аналогичное утверждение.

Пусть $U : \C \to \D$ -- некоторый функтор.
Докажите следующие утверждения:
\begin{enumerate}
\item Если $U$ является правым сопряженным, то он сохраняет мономорфизмы.

Пусть $f : X \to Y$ -- моно. Пусть в следующей диаграмме $U(f) \circ g = U(f) \circ h$. 
\[
\xymatrix{
A \ar@<+0.5ex>[r]^{g} \ar@<-0.5ex>[r]_{h} & U(X) \ar[r]^{U(f)} & U(Y)
}
\]
$\phi^{-1}(U(f) \circ g) = \phi^{-1}(U(f) \circ h)$\\
$f \circ \phi^{-1}(g) = f \circ \phi^{-1}(h) \Rightarrow \phi^{-1}(g) = \phi^{-1}(h) \Rightarrow f = g$

\item Если $U$ является строгим, то обратное верно, то есть если $U(f)$ -- мономорфизм, то $f$ также является мономорфизмом.

\[
\xymatrix{
A \ar@<+0.5ex>[r]^{g} \ar@<-0.5ex>[r]_{h} & X \ar[r]^{f} & Y
}
\]

$f \circ g = f \circ h, ~U $ строгий  $ \Rightarrow U(f \circ g) = U(f \circ h) \Rightarrow U(f) \circ U(g) = U(f) \circ U(h) \Rightarrow U(g) = U(h) \Rightarrow \phi^{-1}(U(g)) = \phi^{-1}(U(h)) \Rightarrow g = h$

\end{enumerate}

\item Докажите, что у забывающего функтора $U : \cat{Cat} \to \cat{Graph}$, сконструированного в 5 ДЗ, существует левый сопряженный.

Построим левый обратный $F$.\\
Пусть $F(~(V, E)~) = C$, где \\
$Ob(C) = V$\\
$Hom(v_a, v_b) = \{[v_a, E(v_a, v_1), v_1, E(v_1, v_2), v_2, ..., v_b] ~|~ v_i \in V\}$ --- произвольные конечные пути
(если $v_a = v_b$, то $[v_a] \in Hom(v_a, v_b)$ -- нейтральный элемент) \\
$F_V(v) = v$\\
$F_E(e^{a\to b}) = [a, e^{a\to b}, b]$\\
Композиция морфизмов --- композиция путей.

Покажем, что $Hom(A, U(B)) \simeq Hom(F(A), B)$. \\
Пусть $f \in Hom(A, U(B))$, тогда ему можно однозначно сопоставить $g \in Hom(F(A), B)$:\\
$g(V) = f(V)$\\
$g([v_a, E(v_a, v_1), v_1, E(v_1, v_2), v_2, ..., v_b]) = f(E(v_a, v_1)) \circ f(E(v_1, v_2)) ...$\\
$g([v_a]) = [f(v_a)]$

\item Докажите, что левый сопряженный к некоторому функтору $U$ уникален с точностью до изоморфизма, то есть если $F \dashv U$ и $F' \dashv U$, то $F \simeq F'$.

Пусть $\alpha_A = \epsilon_{F'A} \circ F\eta'_A, ~~\beta_A = \epsilon'_{FA} \circ F'\eta_A$

Рассмотрим диаграмму
\[
\xymatrix{
F(A) \ar[r]^{\eta_A} \ar @{--<} `l/15pt[d] `[dd]_{\alpha_A} [dd] \ar[d]^{F_{\eta'_A}} & FGF(A) \ar[dr]^{id} \ar[d]^{F{\eta'_{GFA}}} & \\
FGF'(A) \ar[r]_{FGF'\eta_A} \ar[d]^{\epsilon_{F'A}} & FGF'GF(A) \ar[r]_{FG\epsilon'_{FA}} \ar[d]^{\epsilon_{F'GFA}} & FGF(A) \ar[d]^{\epsilon_{FA}} \\
F'(A) \ar @{--<} `d/15pt[r] `[rr]_{\beta_A} [rr]
 \ar[r]_{F'\eta_A} & F'GF(A) \ar[r]_{\epsilon'_{FA}} & F(A)
}
\]
Верхний квадрат коммутирует по натуральности $\eta'$, два нижних -- по натуральности $\epsilon$. Правый верхний треугольник -- так как $F' \dashv U$. Композиция стрелок $F(A)\to FGF(A) \to FGF(A) \to F(A)$ равна $id$, так как $F \dashv U$. Значит $\beta_A \circ \alpha_A = id$. Если построить симметричную диаграмму (меняем $F, F'$), то получится, что $\alpha_A \circ \beta_A = id$. \\
То есть $\alpha$ --- изоморфизм функторов $F, F'$.

\item Есть ли у забывающего функтора $U : \Grp \to \Set$ правый сопряженный? Докажите это.

Рассмотрим копроизведение абелевых групп. Оно равно произведению этих групп. Однако копроизведение множеств --- это размеченное объединение, которое не изоморфно произведению. Раз он не сохраняет копроизведения, то он не левый сопряженный.

\item Есть ли у забывающего функтора $U : \Grp \to \Mon$ правый сопряженный? Докажите это.

Рассмотрим функтор $F : \Mon \to \Grp$, который оставляет только обратимые 
элементы моноида.

Пусть $f: U(A) \to B$. $f(a) * f(a^{-1}) = f(0) = 0 \Rightarrow f(a) = f(a^{-1})^{-1} \Rightarrow f(A) \subset F(B)$. Значит можно смотреть на $f$ как на гомоморфизм групп. В обратную сторону тоже верно. \\
Значит $Hom(U(A), B) \simeq Hom(A, F(B))$, где гомоморфизмы переходят в себя же.


\item Пусть $\cat{rGraph}$ -- категорий рефлексивных графов.
Объекты этой категории -- это графы, в которых для каждой вершины $x$ выбрана петля $id_x$ в этой вершине.
Морфизмы -- морфизмы графов, сохраняющие тождественные петли.

Категория графов в данном упражнении не будет работать, но вместо $\cat{rGraph}$ можно взять категорию малых группоидов или категорию малых категорий; решение при этом не изменится.

Докажите, что у функтора $\Gamma : \cat{rGraph} \to \Set$, сопоставляющего каждому рефлексивному графу множество его вершин, существует правый сопряженный $C : \Set \to \cat{rGraph}$ и левый сопряженный $D : \Set \to \cat{rGraph}$,
и у $D$ существует левый сопряженный $\Pi_0 : \cat{rGraph} \to \Set$.
Таким образом, мы получаем следующую цепочку сопряженных функторов:
\[ \Pi_0 \dashv D \dashv \Gamma \dashv C \]

\item Докажите, что категории $\cat{Fam}_I$ и $\Set/I$ эквивалентны.

Пусть $f, g : I \to Set$ -- объекты $\cat{Fam}_I$, $\mathfrak{f} \in Hom(f, g)$.\\
Определим $F$:\\
$F(f) = (A, fst),$ где $A = \{(i, f(i)) ~|~ i \in I\}$\\
$F(\mathfrak{f}) = \lambda (A, fst). ~(B, fst) ~~:~~F(f)\to F(g), $ где\\
 $B = \{(i, \mathfrak{f} ~i~ f_i) ~|~ (i, f_i) \in A \}$\\
(если смотреть на $\mathfrak{f}$ как на функцию $\prod\limits_{i : I} \to f~i \to g~i$)\\
Коммутативность диаграммы ($fst = fst \circ F(\mathfrak{f})$) очевидна.
\\
Очевидно, что $F(id) = id$.  \\
Если $\mathfrak{g}\in Hom(g, h)$, то $F(\mathfrak{g} \circ \mathfrak{f}) = F(\mathfrak{g}) \circ F(\mathfrak{f})$, так как $\mathfrak{g} ~i~ (\mathfrak{f} ~i~ (f~i)) = (\mathfrak{g} \circ \mathfrak{f}) ~i~(f~i)$.

Таким образом, $F$ -- функтор.\\
$Hom(f, g) \simeq Hom(F(f), F(g))$, так как $Hom(F(f), F(g))$ --- множество функций, которые сохраняют первый элемент пары (так как диаграмма из определения $\Set/I$ коммутирует), а значит оно эквивалентно семейству функций 
$f~i \to g~i$.\\

Получается, что $F$ полный и строгий. Осталось проверить существенную сюръективность.\\

Пусть $w : (A, f: A \to I), ~A\in \Set$. \\
Возьмем $g = f^{-1} : I \to A$, который по индексу возмращает множество прообразов.\\

$F(g) = (B, fst),$ где $B = \{ (i, f^{-1}(i)) ~|~ i\in I\}$\\
Так как $B \simeq A$, то по $fst^{B \to I}$ можно построить $f^{A \to I}$ и наоборот, а значит $F(g) \simeq w$. То есть для $F$ существенно сюръективна. 

Значит $F$ -- экви.

\item Пусть $\C$ -- декартовая категория.
Если $A$ -- объект $\C$, то мы можем определить функтор $A^* : \C \to \C/A$ как $A^*(B) = (A \times B, \pi_1)$ и $A^*(f) = \id_A \times f$.
\begin{enumerate}
\item Докажите, что у $A^*$ есть левый сопряженный.

Возьмем  $(X, f) \in \C/A,~Y \in \C$.\\
$Hom((X, f), A^*(Y))$ состоит из функций $\langle h_1, h_2\rangle : X \to A \times Y$ таких, что $\pi_1 \circ \langle h_1, h_2\rangle = f$. То есть $h_1 = f$, а $h_2 : X \to Y$ -- произвольная\\
Значит $Hom((X, f), A^*(Y)) \simeq Hom(X, Y)$. \\
Тогда можно определить забывающий функтор $F((X, f)) = X$, который будет левым сопряженным. (морфизмы $X \to Y$ переводятся в себя же).

\item Докажите, что если $\C$ декартово замкнута и в $\C$ есть уравнители, то у $A^*$ есть правый сопряженный.
\end{enumerate}
\end{enumerate}

\end{document}
