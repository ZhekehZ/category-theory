\documentclass[draft]{article}
\usepackage[russian]{babel}
\usepackage[utf8]{inputenc}
\usepackage{cmap}
\usepackage{amsfonts}
\usepackage{amssymb}
\usepackage{amsmath}
\usepackage[all]{xy}

\usepackage{hyperref}
\hypersetup{
    colorlinks=true,
    linkcolor=blue,
    filecolor=magenta,      
    urlcolor=cyan,
}
\urlstyle{same}

\newcommand{\cat}[1]{\mathbf{#1}}
\renewcommand{\C}{\cat{C}}
\newcommand{\D}{\cat{D}}
\newcommand{\Set}{\cat{Set}}
\newcommand{\FinSet}{\cat{FinSet}}
\newcommand{\Grp}{\cat{Grp}}
\newcommand{\CMon}{\cat{CMon}}
\newcommand{\Ab}{\cat{Ab}}
\newcommand{\Mat}{\cat{Mat}}
\newcommand{\Num}{\cat{Num}}
\newcommand{\fs}[1]{\mathrm{#1}}
\newcommand{\Ob}{\fs{Ob}}
\newcommand{\Hom}{\fs{Hom}}


\newenvironment{tolerant}[1]{\par\tolerance=#1\relax}{\par}

\begin{document}

\title{Задания}
\maketitle

\begin{enumerate}

\item Пусть $\C$ -- категория предпорядка, а $\D$ -- нет.
\begin{enumerate}
\item Могут ли $\C$ и $\D$ быть изоморфны?
\item Могут ли $\C$ и $\D$ быть эквивалентны?
\end{enumerate}

\item Пусть $\C$ -- категория с одним объектом, а $\D$ -- нет.
\begin{enumerate}
\item Могут ли $\C$ и $\D$ быть изоморфны?
\item Могут ли $\C$ и $\D$ быть эквивалентны?
\end{enumerate}

\item Пусть $\C$ -- дискретная категория, а $\D$ -- нет.
\begin{enumerate}
\item Могут ли $\C$ и $\D$ быть изоморфны?
\item Могут ли $\C$ и $\D$ быть эквивалентны?
\end{enumerate}

\item Пусть $\C$ -- группоид, а $\D$ -- нет.
\begin{enumerate}
\item Могут ли $\C$ и $\D$ быть изоморфны?
\item Могут ли $\C$ и $\D$ быть эквивалентны?
\end{enumerate}

\item Докажите, что $\Num$ эквивалентна $\FinSet$. Изоморфны ли эти категории?

Рассмотрим функтор
$ F : \Num \to \FinSet$\\
$F(n) = \{1, 2, ... n\} =: A_n$\\
$F((a_1, ..., a_n)) = \lambda x. ~case ~ x ~ of \{ i \Rightarrow a_i \} ~: A_n \to A_k$\\

Так как $|Hom_{\Num}(n, k)| = k^n = |Hom_\FinSet(\{1,...,n\}, \{1,...,k\})|$ и  $F: Hom_{\Num}(n, k) \to Hom_\FinSet(\{1,...,n\}, \{1,...,k\})$ --- инъекция, то $F$ --- сюръекция. Значит $F$ строгий и полный.\\

$\forall ~S \in \Set ~\exists A_{|S|} \simeq S$, так как равномощные множества изоморфны. Значит $\forall F(|S|)\simeq S$. F существенно сюръективен.\\
Получается, что F --- экви.\\

$\FinSet$ не изоморфен $\Num$, так как первый состоит из континуального множества объектов, а второй --- из счетного.

\item Докажите, что $\Mat$ эквивалентна $\Mat^{op}$. Изоморфны ли эти категории?

\item Докажите, что $\FinSet$ не эквивалентна $\Set$.

Пусть $F: \Set \to \FinSet$ --- экви. \\
Тогда $|Hom_\Set(\mathbb{N}, \{0\})| = |Hom_\FinSet(F(\mathbb{N}), F(\{0\}))| < \infty$, что неверно. 

\item Пусть $F, G : \C \to \D$ -- пара функторов.
Естественное преобразование $\alpha : F \to G$ называется \emph{естественным изоморфизмом}, если для любого объекта $X$ в $\C$ морфизм $\alpha_X : F(X) \to G(X)$ является изоморфизмом.

Докажите, что $\alpha : F \to G$ -- естественный изоморфизм тогда и только тогда, когда $\alpha$ -- изоморфизм в категории $\D^\C$.

\[
\xymatrix{
	F(X) \ar@<+0.5ex>[r]^{\alpha_X} \ar[d]_{F(f)} & G(X) \ar@<+0.5ex>[l]^{\beta_X}\ar[d]^{G(f)}\\
	F(X) \ar@<+0.5ex>[r]^{\alpha_Y} & G(Y) \ar@<+0.5ex>[l]^{\beta_Y}
}
\]

Если $\forall X: ~\alpha_X$ -- изо, то позьмем $\beta_X := \alpha_X^{-1}$. Для такого $\beta$ диаграмма выше коммутирует, значит $\beta$ -- естественное проеобразование. Кроме того, $(\alpha \circ \beta)_X = \alpha_X \circ \beta_X = id_{G(X)}$, а $(\beta\circ \alpha )_X = \beta_X \circ \alpha_X = id_{F(X)}$. Значит $\beta = \alpha^{-1}$ в $\D^\C$.\\

Если $\beta = \alpha^{-1}$ в $\D^\C$, то $(\alpha \circ \beta)_X = \alpha_X \circ \beta_X = id_{G(X)}$ (и симметрично с другой стороны). Значит $\forall X:~ \beta_X = \alpha_X^{-1}$, то есть $\forall X: ~\alpha_X$ -- изо.


\item Пусть $\C$ -- декартова категория. Докажите, что функтор $- \times 1$ изоморфен тождественному функтору в $\C^\C$.

\[
\xymatrix{
	X \times 1 \ar@<+0.5ex>[r]^{\pi_1} \ar[d]_{\langle f, !\rangle} & X \ar@<+0.5ex>[l]^{\langle id, !\rangle}\ar[d]^{\langle f, !\rangle}\\
	Y \times 1 \ar@<+0.5ex>[r]^{\pi_1} & X \ar@<+0.5ex>[l]^{\langle id, !\rangle}
}
\]

Поскольку диаграмма выше коммутирует, то $\pi_1 : - \times 1 \to id$ и $\langle id, !\rangle$ --- естественные преобразования, причем взаимно обратные. То есть $\pi_1$ --- изоморфизм этих функторов.

\newpage
\item Пусть $\pmb{\rightrightarrows}$ -- категория, состоящая из двух объектов $\{ v, e \}$ и четырех морфизмов $\{ id_v : v \to v, id_e : e \to e, d : v \to e, c : v \to e \}$.
Докажите, что категории $\cat{Graph}$ (эта категория определяется в предыдущем ДЗ) и $\Set^{\pmb{\rightrightarrows}^{op}}$ эквивалентны.
Изоморфны ли эти категории?

Рассмотрим функтор $F: \cat{Graph} \to \Set^{\pmb{\rightrightarrows}^{op}}$.\\
$F(~(V, E)~) = g$, где g --- функтор $\pmb{\rightrightarrows}^{op} \to \Set$ определененный следующим образом:\\
$g(v) := V$\\
$g(e) := \{(x, y, e) ~|~ x, y \in V, ~e \in E(x, y)\}$\\
$g(c^{op}) := \pi_1$\\
$g(d^{op}) := \pi_2$\\
$g(id) := id$\\

Далее $g(e)$ буду обозначать как $(x, y, Exy)$

Пусть $(f_V, f_E)$ -- морфизм графов. Тогда должно быть $F(~(f_V, f_E)~) = \alpha$, где $\alpha$ -- какое-то Е.П. функторов $\pmb{\rightrightarrows}^{op} \to \Set$. \\

Пусть $\alpha_v = f_V, ~~\alpha_e = \langle f_V, f_V, f_E\rangle$. Докажем, что $\alpha$ -- Е.П. Пусть $F(~(V, E)~) = g_1, ~F(~(f_V(V), f_E(E))~) = g_2$ тогда $\alpha: g_1 \to g_2$. Так как в $\pmb{\rightrightarrows}^{op}$ все стрелки направлены от $e$ к $v$, достаточно рассмотреть диаграмму

\[
\xymatrix{
g_1(e) \ar[r]^{\langle f_V, f_V, f_E\rangle} \ar[d]_{\pi_i} & g_2(e) \ar[d]_{\pi_i}\\
g_1(v) \ar[r]^{f_V} & g_2(v)
}
\]

Или точнее
\[
\xymatrix{
(x,y,Exy) \ar[rr]^{\langle f_V, f_V, f_E\rangle~~~~~~~~~} \ar[d]_{\pi_i} & & (f_V(x),f_V(y),f_E(Exy)) \ar[d]_{\pi_i}\\
V \ar[rr]^{f_V} & & f_V(V)
}
\]

Так как данная диаграмма коммутирует для $\pi_{[1, 2]}$, $\alpha$ -- Е.П.

Кроме того, если применить $F$ к композиции морфизмов $(f_V^1, f_E^1) \circ (f_V^2, f_E^2) = (f_V^1 \circ f_V^2, f_E^1\circ f_E^2)$, то полученное естественное преобразование будет в точности композицией естественных преобразований (нужно просто к диаграмме выше добавить еще один квадрат справа). То есть $F(x\circ y) = F(x)\circ F(y)$. Значит $F$ --- корректный функтор.

Очевидно, что есть биекция между парой $(f_V, f_E)$ и парой $(\alpha_v, \alpha_e)$. Значит $Hom(X, Y) \simeq Hom(F(X), F(Y))$. То есть $F$ --- строгий и полный.


Пусть $f: \pmb{\rightrightarrows}^{op} \to \Set$ -- некоторый функтор. Возьмем граф $(V, E)$, где \\
$V = f(v)$\\
$E(x, y) = \{e ~|~ e \in f(e),~ f(c^{op})(e) = x,~ f(d^{op})(e) = y\}$\\

Тогда $F(~(V, E)~)$ будет равен $g$\\
$g(v) := V = f(v)$\\
$g(e) := \{(f(c^{op})(e), ~f(d^{op})(e), ~e) ~|~ e \in f(e)\}$\\
$g(c^{op}) := \pi_1$\\
$g(d^{op}) := \pi_2$\\
$g(id) := id$\\

Рассмотрим пару естественных преобразований $\alpha, \beta$, где $\alpha_v  = \beta_v = id,~\alpha_e = \langle f(c^{op}),~f(d^{op}), id \rangle, ~\beta_e = \pi_3$. Так как $\alpha_v \circ \beta_v = id,~~\beta_v \circ \alpha_v = id,~~\alpha_e \circ \beta_e = id,~~\beta_e \circ \alpha_e = id$, данные преобразования являются изоморфизмами. То есть $F(~(V, E)~) \simeq f$. Значит $F$ существенно сюръективен.\\

Таким образом, $F$ --- экви.

Про изоморфность надо еще подумать.

\item Пусть $\D$ -- рефлективная подкатегория $\C$.
\begin{enumerate}
\item Докажите, что рефлектор $\Ob(\C) \to \Ob(\D)$ является фнуктором $R : \C \to \D$.\\

Определим действие $R$ на морфизмах. Пусть $f: X \to Y$. Тогда $R(f) = f'$ из диаграммы ниже.
\[
\xymatrix{
X \ar[r]^f \ar[d]^{\eta_X} & Y \ar[d]^{\eta_Y}\\
R(X) \ar@{-->}[r]^{\exists! f'} & R(Y)
}
\]

$R(id) = id:$
\[
\xymatrix{
X \ar[r]^{id} \ar[d]_{\eta_X} & X \ar[d]_{\eta_X} \\
R(X) \ar@{-->}[r]^{\exists! h = id} & R(X)
}
\]
$R(f\circ g) = R(f) \circ R(g):$
\[
\xymatrix{
X \ar[r]^g \ar[d]_{\eta_X} & Y\ar[r]^f \ar[d]_{\eta_Y} & Z \ar[d]_{\eta_Z}\\
R(X) \ar@{-->}[r]^{\exists!g'} 
\ar@<1ex>@{-->}`d/10pt`[rr]_{\exists!h}
& R(Y) \ar@{-->}[r]^{\exists!f'} & R(Z)
}
\]

\item Докажите, что $\eta$ является естественным преобразованием между $\fs{Id}_\C$ и $i \circ R$, где $i : \D \to \C$ -- функтор вложения.

Следующая диаграмма коммутирует и является определением для Е.П между $\fs{Id}_\C$ и $i \circ R$:
\[
\xymatrix{
X \ar[r]^{i \circ \eta_X} \ar[d]_{f} & i(R(X)) \ar@{-->}[d]^{\exists!h =(i \circ R)(f)} \\
Y \ar[r]^{i \circ \eta_Y} &  i(R(Y))
}
\]
\end{enumerate}

\item Пусть $F : \CMon \to \Ab$ -- рефлектор вложения $i : \Ab \to \CMon$.
\begin{enumerate}
\item \tolerant{500}{ Приведите пример конечного нетривиального коммутативного моноида $X$, такого что $|F(X)| = |X|$. }

Можно взять любую нетривиальную абелеву группу (ее вложение в $\CMon$).
\item Приведите пример конечного коммутативного моноида $X$, такого что $|F(X)| < |X|$.

Рассмотрим моноид $M = \{0, 1\}$ с операцией $max$.\\
Докажем, что $F(M) = \{0\}$. Пусть $f : M \to G$. \\
Тогда $f(0) = 0,~f(1) = f(1 ~`max`~ 1) = f(1)\ast f(1)$\\
$0 = f(0) = f(1) \ast f(1)^{-1} = f(1) \ast  f(1) \ast f(1)^{-1} = f(1)$. \\
То есть $f(M) = \{0\}$. Тогда единственный существующий гомоморфизм $F(M) \to G$ заставит коммутировать диаграмму из определения рефлектора.\\

$|M| = 2, ~|F(M)| = 1$


\item Приведите пример коммутативного моноида $X$, такого что $\eta_X : X \to i(F(X))$ -- не сюръективна.\\

$X = (\mathbb{N}, +)$\\
$F(X) = (\mathbb{Z}, +)$\\
$i(F(X)) = (\mathbb{Z}, +)$\\
$\eta_X(x) = x$ -- не сюръекция

\item Докажите, что для любого конечного коммутативного моноида $X$ функция $\eta_X : X \to i(F(X))$ является сюръективной. В частности $|F(X)| \leq |X|$.\\

Пусть дан некоторый коммутативный моноид $M$. Возьмем группу Гротендика $G(M)$, соответствующую этому моноиду.\\
(нашел вот тут \url{https://en.wikipedia.org/wiki/Grothendieck_group}, решил не переписывать определение).

Докажем, что $M \to i(G(M))$ --- сюръекция.\\
Пусть $(a, b) \in G(M)$. \\
Если $\exists b^{-1}$, то $(a, b) \sim (a*b^{-1}, 0) = G(a*b^{-1})$.\\
Если это не так, то, поскольку моноид конечный, $\exists n,m \in \mathbb{N}, n > m: ~b^n=b^m$. Тогда $(a, b) \sim (a*b^{m+1}, b^{m+1}) = (a*b^{n+1}, b^{m+1}) = (a*b^{m+1}*b^{n - m}, b^{m+1}) \sim (a*b^{n - m}, 0) = G(a*b^{n - m})$.\\

Таким образом, $i \circ G$ --- сюръекция, а значит $\eta_X$ --- сюръекция, так как $F$ изоморфен $G$.
\end{enumerate}

\end{enumerate}

\end{document}
