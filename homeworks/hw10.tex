\documentclass[draft]{article}
\usepackage[russian]{babel}
\usepackage[utf8]{inputenc}
\usepackage{cmap}
\usepackage{amsfonts}
\usepackage{amssymb}
\usepackage{amsmath}
\usepackage[all]{xy}
\usepackage{stmaryrd}
\usepackage{bussproofs}
\usepackage{turnstile}
\usepackage{mathtools}

\renewcommand{\turnstile}[6][s]
    {\ifthenelse{\equal{#1}{d}}
        {\sbox{\first}{$\displaystyle{#4}$}
        \sbox{\second}{$\displaystyle{#5}$}}{}
    \ifthenelse{\equal{#1}{t}}
        {\sbox{\first}{$\textstyle{#4}$}
        \sbox{\second}{$\textstyle{#5}$}}{}
    \ifthenelse{\equal{#1}{s}}
        {\sbox{\first}{$\scriptstyle{#4}$}
        \sbox{\second}{$\scriptstyle{#5}$}}{}
    \ifthenelse{\equal{#1}{ss}}
        {\sbox{\first}{$\scriptscriptstyle{#4}$}
        \sbox{\second}{$\scriptscriptstyle{#5}$}}{}
    \setlength{\dashthickness}{0.111ex}
    \setlength{\ddashthickness}{0.35ex}
    \setlength{\leasturnstilewidth}{2em}
    \setlength{\extrawidth}{0.2em}
    \ifthenelse{%
      \equal{#3}{n}}{\setlength{\tinyverdistance}{0ex}}{}
    \ifthenelse{%
      \equal{#3}{s}}{\setlength{\tinyverdistance}{0.5\dashthickness}}{}
    \ifthenelse{%
      \equal{#3}{d}}{\setlength{\tinyverdistance}{0.5\ddashthickness}
        \addtolength{\tinyverdistance}{\dashthickness}}{}
    \ifthenelse{%
      \equal{#3}{t}}{\setlength{\tinyverdistance}{1.5\dashthickness}
        \addtolength{\tinyverdistance}{\ddashthickness}}{}
        \setlength{\verdistance}{0.4ex}
        \settoheight{\lengthvar}{\usebox{\first}}
        \setlength{\raisedown}{-\lengthvar}
        \addtolength{\raisedown}{-\tinyverdistance}
        \addtolength{\raisedown}{-\verdistance}
        \settodepth{\raiseup}{\usebox{\second}}
        \addtolength{\raiseup}{\tinyverdistance}
        \addtolength{\raiseup}{\verdistance}
        \setlength{\lift}{0.8ex}
        \settowidth{\firstwidth}{\usebox{\first}}
        \settowidth{\secondwidth}{\usebox{\second}}
        \ifthenelse{\lengthtest{\firstwidth = 0ex}
            \and
            \lengthtest{\secondwidth = 0ex}}
                {\setlength{\turnstilewidth}{\leasturnstilewidth}}
                {\setlength{\turnstilewidth}{2\extrawidth}
        \ifthenelse{\lengthtest{\firstwidth < \secondwidth}}
            {\addtolength{\turnstilewidth}{\secondwidth}}
            {\addtolength{\turnstilewidth}{\firstwidth}}}
        \ifthenelse{\lengthtest{\turnstilewidth < \leasturnstilewidth}}{\setlength{\turnstilewidth}{\leasturnstilewidth}}{}
    \setlength{\turnstileheight}{1.5ex}
    \sbox{\turnstilebox}
    {\raisebox{\lift}{\ensuremath{
        \makever{#2}{\dashthickness}{\turnstileheight}{\ddashthickness}
        \makehor{#3}{\dashthickness}{\turnstilewidth}{\ddashthickness}
        \hspace{-\turnstilewidth}
        \raisebox{\raisedown}
        {\makebox[\turnstilewidth]{\usebox{\first}}}
            \hspace{-\turnstilewidth}
            \raisebox{\raiseup}
            {\makebox[\turnstilewidth]{\usebox{\second}}}
        \makever{#6}{\dashthickness}{\turnstileheight}{\ddashthickness}}}}
        \mathrel{\usebox{\turnstilebox}}}

\newcommand{\cat}[1]{\mathbf{#1}}
\renewcommand{\C}{\cat{C}}
\newcommand{\D}{\cat{D}}
\newcommand{\y}{\cat{y}}
\newcommand{\Set}{\cat{Set}}
\newcommand{\Grp}{\cat{Grp}}
\newcommand{\Ab}{\cat{Ab}}
\newcommand{\Mat}{\cat{Mat}}
\newcommand{\Num}{\cat{Num}}
\newcommand{\red}{\Rightarrow}
\renewcommand{\ll}{\llbracket}
\newcommand{\rr}{\rrbracket}

\newcommand{\pb}[1][dr]{\save*!/#1-1.2pc/#1:(-1,1)@^{|-}\restore}
\newcommand{\po}[1][dr]{\save*!/#1+1.2pc/#1:(1,-1)@^{|-}\restore}

\newcommand{\im}{\mathrm{Im}}
\newcommand{\bool}{\mathrm{Bool}}
\newcommand{\true}{\mathrm{true}}
\newcommand{\false}{\mathrm{false}}
\newcommand{\andb}{\mathrm{and}}
\newcommand{\orb}{\mathrm{or}}
\newcommand{\inj}{\mathrm{inj}}

\newcommand{\ev}{\mathrm{ev}}
\newcommand{\zero}{\mathrm{zero}}
\newcommand{\suc}{\mathrm{suc}}
\newcommand{\rec}{\mathrm{rec}}

\newenvironment{tolerant}[1]{\par\tolerance=#1\relax}{\par}

\begin{document}

\title{Задания}
\maketitle

\begin{enumerate}

\item Докажите, что если мы добавим в лямбда исчисление тип натуральных чисел $\mathbb{N}$ с термами и аксиомами, приведенными ниже, то такое лямбда исчесление можно проинтерпретировать в любой декартово замкнутой категории с объектом натуральных чисел.
\begin{center}
\AxiomC{}
\UnaryInfC{$\Gamma \vdash \zero : \mathbb{N}$}
\DisplayProof
\quad
\AxiomC{$\Gamma \vdash n : \mathbb{N}$}
\UnaryInfC{$\Gamma \vdash \suc(n) : \mathbb{N}$}
\DisplayProof
\end{center}

\begin{center}
\AxiomC{$\Gamma \vdash z : D$}
\AxiomC{$\Gamma, x : \mathbb{N}, r : D \vdash s : D$}
\AxiomC{$\Gamma \vdash n : \mathbb{N}$}
\TrinaryInfC{$\Gamma \vdash \rec(z, s, n) : D$}
\DisplayProof
\end{center}

\begin{center}
\AxiomC{$\Gamma \vdash z : D$}
\AxiomC{$\Gamma, x : \mathbb{N}, r : D \vdash s : D$}
\BinaryInfC{$\Gamma \vdash \rec(z, s, \zero) \equiv z : D$}
\DisplayProof
\end{center}

\begin{center}
\AxiomC{$\Gamma \vdash z : D$}
\AxiomC{$\Gamma, x : \mathbb{N}, r : D \vdash s : D$}
\AxiomC{$\Gamma \vdash n : \mathbb{N}$}
\TrinaryInfC{$\Gamma \vdash \rec(z, s, \suc(n)) \equiv s[x := n, r := \rec(z, s, n)] : D$}
\DisplayProof
\end{center}

\item Определите структуру монады на функторе $\mathrm{Term}_\Sigma$ для любой сигнатуры $\Sigma$.

$\eta = id' : 1 \to T $ --- естественное преобразование, которое возвращает функтор, переводящий переменную в терм из одной переменной

$\mu = id'' : T\circ T \to T$ --- естественное преобразование, которое интерпретирует терм с переменными-термами как один терм

Поскольку $\mu, \eta$ не влияют на структуру терма, диаграммы из определения монады для них коммутируют

\newpage
\item Определите регулярную теорию, моделями которой являются малые категории.

$\mathcal{S} = \{a, h\}$\\
$\mathcal{F} = \{src: h \to a, ~ dst: h \to a,~id:~a \to h\}$\\
$\mathcal{P} = \{comp: h\times h \times h\}$\\
$\Sigma = \{\mathcal{S}, \mathcal{F}, \mathcal{P}\}$\\
$\mathcal{A}:$\\
$
\begin{matrix*}[l]
comp(f, g, t) \sststile{}{f,g,t:h} src~f=dst~g ~\land~ dst~f = dst~t~\land~src~g=src~t\\\\
src~f=dst~g  \sststile{}{f,g:h} \exists(t:h)comp(f, g, t)\\\\
comp(f, g, t_1)\land comp(f, g, t_2)  \sststile{}{f,g,t_1,t_2:h} t_1=t_2\\\\
comp(f, g, t_1) \land comp(t_1, r, w_1)\land\\~~~~~~~~\land
comp(g, r, t_2) \land comp(f, t_2, w_2)
 \sststile{}{f,g,r,t_1,t_2,w_1,w_2:h} w_1=w_2\\\\
\top \sststile{}{x:a} src~(id~x)=x ~\land ~dst(id~x) = x\\\\
comp(f, id~x, g) \sststile{}{x:a, f,g:h} f = g  \\\\
comp(id~x, f, g) \sststile{}{x:a, f,g:h} f = g  \\\\
\end{matrix*}
$

$\mathcal{T} = \{\Sigma, \mathcal{A}\}$ -- регулярная теория

\item Опишите интерпретацию импликации, кванторов и равенства в $\Set$.

$
\begin{matrix*}[l]
\llbracket a \to b \rrbracket = \llbracket \neg a \lor b \rrbracket \\\\
\llbracket a = b \rrbracket = \llbracket (a \to b) \land (b \to a)\rrbracket\\\\
\llbracket \forall a~\phi(a) \rrbracket = \bigcap\limits_a \llbracket \phi(a) \rrbracket\\\\
\llbracket \exists a~\phi(a) \rrbracket = \llbracket \neg (\forall a~\neg \phi(a)) \rrbracket
\end{matrix*}
$


\item Докажите корректность следующего правила вывода
\begin{center}
\AxiomC{$\varphi \sststile{}{V} a = b$}
\AxiomC{$\varphi \sststile{}{V} \psi[x := a]$}
\BinaryInfC{$\varphi \sststile{}{V} \psi[x := b]$}
\DisplayProof
\end{center}

%\[
%\xymatrix{
%A \ar@{^{(}->}[rr] \ar@{^{(}->}[d] &&
%X \ar@{^{(}->}[rr] \ar@{^{(}->}[d] && d_\phi \ar@{^{(}->}[d]^{\llbracket \phi \rrbracket} \\
%B \ar@{^{(}->}[rr]_{\llbracket a = b \rrbracket} && \llbracket V \rrbracket  \ar@{^{(}->}[rr]_{\langle id, \llbracket a \rrbracket\rangle} && \llbracket V \rrbracket  \times \llbracket s \rrbracket 
%}
%\]



\end{enumerate}

\end{document}
