\documentclass[draft]{article}
\usepackage[russian]{babel}
\usepackage[utf8]{inputenc}
\usepackage{cmap}
\usepackage{amsfonts}
\usepackage[all]{xy}

\newcommand{\cat}[1]{\mathbf{#1}}
\renewcommand{\C}{\cat{C}}
\newcommand{\Set}{\cat{Set}}
\newcommand{\Grp}{\cat{Grp}}
\newcommand{\Ab}{\cat{Ab}}
\newcommand{\Vec}{\cat{Vec}}
\newcommand{\Hask}{\cat{Hask}}
\newcommand{\Mat}{\cat{Mat}}
\newcommand{\Num}{\cat{Num}}

\newcommand{\im}{\mathrm{Im}}
\newcommand{\bool}{\mathrm{Bool}}
\newcommand{\true}{\mathrm{true}}
\newcommand{\false}{\mathrm{false}}
\newcommand{\andb}{\mathrm{and}}
\newcommand{\orb}{\mathrm{or}}

\begin{document}

\title{Задания}
\maketitle

\begin{enumerate}

\item Опишите в категории (пред)порядка следующие конструкции:
\begin{enumerate}
\item Начальные объекты.
\item Копроизведения объектов.
\end{enumerate}

\item Докажите следующие факты про пулбэки мономорфизмов:
\begin{enumerate}
\item Докажите, что пулбэк мономорфизма также является мономорфизмом.
\item Докажите, что пулбэк регулярного мономорфизма также является регулярным мономорфизмом.
\end{enumerate}

\item Докажите следующие факты про пулбэки эпиморфизмов:
\begin{enumerate}
\item Докажите, что пулбэк сюръективной функции в $\Set$ также является сюръективной функцией.
\item Докажите, что предыдущее утверждение не верно в категории моноидов для эпиморфизмов. Другими словами, необходимо привести пример эпиморфизма в категории моноидов, некоторый пулбэк которого не является эпиморфизмом.
\end{enumerate}

\item Докажите, что если $A \amalg B$ существует, то $B \amalg A$ тоже существует и изоморфен $A \amalg B$.

\item Начальный объект $0$ произвольной категории называется \emph{строгим}, если любой морфизм вида $X \to 0$ является изоморфизмом.
Например, в $\Set$ пустое множество является строгим начальным объектом.
В $\Grp$ тривиальная группа не является строгим начальным объектом, хоть и является начальным.

Докажите, что в произвольной категории начальный объект $0$ является строгим тогда и только тогда, когда для любого $X$ произведение $X \times 0$ существует и $X \times 0 \simeq 0$.

\begin{itemize}
\item[$\Rightarrow$)] 0 -- строгий.\\
Рассмотрим диаграмму 
\[ \xymatrix{ 
& A \ar[dl]_f \ar[dr]^g \ar@{-->}[d]^k &\\
X & 0 \ar[l]^{\pi_1}\ar[r]_{\pi_2} & 0
} \]
Так как 0 строгий, то $g, \pi_2$ --- изо. Значит $\exists k = \pi_2^{-1} \circ g$. Так как $k$ --- изо, а 0 начальный, то $k^{-1}$ --- уникальный. Значит $k$ --- уникальный. Так как 0 --- начальный объект, то $\pi_1$ --- уникальный, а значит $\pi_1 \circ k$ --- уникальный. Значит $f = \pi_1 \circ k$. Таким образом, вся диаграмма коммутирует и $k$ --- уникальный.
\item[$\Leftrightarrow$)] $X\times 0 \simeq 0$.\\
Пусть $f : A \to 0$. Рассмотрим $A \times 0$
\[ \xymatrix{ 
& A \ar[dl]_{id} \ar[dr]^f \ar@{-->}[d]^{\exists!k} &\\
A & 0 \ar[l]^{\pi}\ar[r]_{id} & 0
} \]
$f = k ~~\land~~ \pi\circ k = id \Rightarrow \pi\circ f = id$\\
$f \circ \pi = id$, так как это стрелка из начального объекта.
\end{itemize}


\item Пусть в диаграмме вида
\[ \xymatrix{ \bullet \ar[r] \ar[d] & \bullet \ar[r] \ar[d] & \bullet \ar[d] \\
              \bullet \ar[r]        & \bullet \ar[r]        & \bullet
            } \]
правый квадрат является пулбэком.
Докажите, что левый квадрат является пулбэком тогда и только тогда, когда внешний прямоугольник является пулбэком.

\begin{itemize}
\item[$\Rightarrow)$] Левый квадрат --- пулбэк.\\

\[ \xymatrix{ 
A \ar@{..>}[dr] \ar[ddr]_f \ar[rrrd]^g \ar@{..>}[rrd] & & & &\\
& \bullet_{(a)} \ar[r]_{ab} \ar[d]^a & \bullet_{(b)} \ar[r] \ar[d] & \bullet_{(c)} \ar[d]\\
& \bullet \ar[r]        & \bullet \ar[r]        & \bullet
} \]

Пусть есть $A, f, g$. Так как правый квадрат --- пулбэк, то $\exists! h : A \to \bullet_{(b)}$ такой, что диаграмма коммутирует. Так как левый квадрат является пулбэком, то $\exists! k : A \to \bullet_{(a)}$ такой, что левая половина диаграммы коммутирует. Если существует $k' : A \to \bullet_{(a)}$, при котором коммутативен подграф, состоящий из него, внешнего прямоугольника и $f, g$. Но тогда $ab \circ k' = h$, так как $h$ --- единственный, для которого соответстующий подграф коммутативен. Но раз так, то $k = k'$, так как с $k', a, f$ коммутирует левый пулбэк.\\
Получается, что для  $\forall f, g ~~\exists!k$ такой, что внешний прямоугольник коммутирует, значит прямоугольник --- пулбэк.
\item[$\Leftrightarrow)$] Внешний прямоугольник--- пулбэк.\\

\[ \xymatrix{ 
A \ar@{..>}[dr] \ar[ddr]_f \ar[rrd]^g & & & &\\
& \bullet_{(a)} \ar[r]_{ab} \ar[d]^a & \bullet_{(b)} \ar[r]_{bc} \ar[d] & \bullet_{(c)} \ar[d]\\
& \bullet \ar[r]      & \bullet \ar[r]        & \bullet
} \]
Пусть есть $f, g$, с которыми левый квадрат коммутативен. Тогда и вся диаграмма коммутативна, а значит $\exists!k : A \to \bullet_{(a)}$, для которого внешний прямоугольник коммутативен. Так как правый квадрат --- пуллбэк, то $g$ --- единственный, для которого вся диаграмма коммутирует. Значит $ab \circ k = g$. Но тогда c $k$ коммутирует и левый квадрат. Пусть  $\exists k'$, обладающий теми же свойствами, что и $k$. Но тогда $bc\circ ab \circ k', f$ подставим в определение для пулбэка для внешнего прямоугольника и получим, что $k = k'$.


\end{itemize}

\item Пусть $f : A \to B$ и $g : B \to C$ -- морфизмы в некоторой категории, а $D \hookrightarrow C$ -- некоторый подобъект $C$.
Докажите, что $(g \circ f)^{-1}(D) \simeq f^{-1}(g^{-1}(D))$.

Пусть $h$ --- вложение $D\to C$. Нужно доказать, что $(g \circ f)^{-1} \circ h = f^{-1} \circ g^{-1} \circ h$. Так как $h$ --- моно, это верно.

\item Докажите, что в $\Ab$ существуют все копроизведения.\\

\[ \xymatrix{ 
A\ar[r]^{\pi_A} \ar[dr]_{f} & A+B \ar@{-->}[d]^k & B \ar[l]_{\pi_B} \ar[ld]^{g}\\
 & C&
} \]
$A+B = (A\times e_B \cup e_A\times B, \langle \ast_A, \ast_B\rangle)$\\
$\pi_A(a) = (a, e_B),~~\pi_B(b) = (e_A, b)$

$\forall f : A \to C, \forall g : B \to C:$\\
$k((a, b)) := f(a)\ast_C g(b) ~~:~A+B \to C$\\

Единственность:\\
Пусть $\exists k': A+B\to C$.\\
$k' \circ \pi_B = g$\\
Любая функция $A+B\to C$ будет иметь вид $(a, b)\mapsto f'(a)\ast g'(b)$ для некоторых $f, g$. \\
Тогда $\forall b \in B:~~ g(b) = (k' \circ \pi_B)(b) = k'((e_A, b)) = f'(e_A)\ast g'(b) = g'(b)~~\Rightarrow g = g'$.\\
Аналогично $f = f'$. Значит $k=k'$.

\item Приведите нетривиальный пример категории, в которой для всех $A$ и $B$ существуют сумма и произведение и $A \amalg B \simeq A \times B$.

Категория предпордка, в которой $\leq$ есть отношение эквивалентности.

\item Идемпотентный морфизм $h : B \to B$ является расщепленным, если существуют $f : A \to B$ и $g : B \to A$ такие, что $g \circ f = id_A$ и $f \circ g = h$.
Докажите, что если в категории существуют коуравнители, то любой идемпотентный морфизм расщеплен.

~~Как было показано ранее, если в категории есть уравнители, то идемпотентный морфизм расщеплен. Если в категории есть уравнители, то в 
$C^{op}$ любой идемпотентный морфизм расщеплен. Так как $Hom_{C^{op}}(X, Y)$ состоит из тех же элементов, что и $Hom(Y, X)$, то соответствующие $f, g$ будут стрелками в исходной категории и для них будут выполняться необходимые свойства. Значит $р$ будет расщепленным.

\item Докажите, что если в категории существуют терминальный объект и пулбэки, то в ней существуют все конечные пределы.

Пусть есть какая-то диаграмма. Рассмотрим произвольную стрелку $f : A \to B$. Существует пулбэк
\[ \xymatrix{ 
A\times_B B \ar[d] \ar[r] & B \ar[d]^{id} \\
A \ar[r]_{f} & B
} \]
Построим такие пулбэки для всех стрелок. Рассмотрим произвольную компоненту слабой связности. Пусть есть 2 смежные стрелки (направление значения не имеет).
\[ \xymatrix{ 
A\ar[r] & B \ar[r] & C
} \]
Рассмотрим также уже построенные для этих стрелок пулбэки
\[ \xymatrix{ 
& A \times_B B \ar[dl] \ar[d] & B \times_C C \ar[dl] \ar[d] \\
A\ar[r] & B \ar[r] & C
} \]
Построим пулбэк для $A \times_B B, B, B \times_C C$
\[ \xymatrix{ 
&& ABC\ar[dl] \ar[d]\\
& A \times_B B \ar[dl] \ar[d] & B \times_C C \ar[dl] \ar[d] \\
A\ar[r] & B \ar[r] & C
} \]
Таким образом, для каждой помпоненты слабой связности можно построить конус.
Теперь объединим конусы для разных компонент с помощью следующего пулбэка
\[ \xymatrix{ 
&C\ar[dl] \ar[dr] &\\
C_1 \ar[dr] && C_2 \ar[dl] \\
& 1 & 
} \]
В результате получим конус $C$. \\
Докажем, что это если сузествует другой конус $C'$ и стрелка $C'\to C$, то она единственна. \\
Возьмем случайный объект А. Рассмотрим произвольный пулбэк, в котором он содержится. 
\[ \xymatrix{ 
C' \ar[dr] \ar[dddrr] \ar[ddrrr]& & &\\
& C \ar[dr] & & &\\
& & A\times_? ? \ar[d] \ar[r] & ? \ar[d] \\
& & A \ar[r] & ?
} \]
Стрелка $C'\to C \to A\times_? ?$ заставляет коммутировать оба трекгольника и уникальна по определению пулбэка. Такие же условия можно написать для остальных объектов. Поскольку $C \to A\times_? ?$ для всех объектов фиксирован, то получаем, что существует единственная стрелка $C'\to C$, что все треугольники одновременно коммутируют.

Осталось показать, что для любого конуса стрелка $C' \to C$ существует.
Рассмотрим произвольную стрелку (если есть) $A \to B$. Для нее есть пулбэк $A\times_B B$. Значит существует стрелка из конуса в объект пулбэка такая, что два треугольника (из определения пулбэка) коммутируют. Таким образом, существуют стреки во все пулбэки первого уровня (те, что непосредственно связаны с исходной диаграммой). На этих пулбэках были также построены пулбэки (второго уровня). Значит  в ких тоже существуют стрелки. Таким образом, по индукции доходим до $C$.

Получили, что существует конус $C$ такой, что для любого другого конуса есть единственная стрелка такая, что треугольники из определения предела коммутируют. Значит $C$ есть предел.

\item Пусть $J = (V,E)$ -- некоторый граф, $D$ -- диграмма формы $J$ в категории $\C$, и $A$ -- конус диаграммы $D$.
Мы будем говорить, что конус $A$ является \emph{слабым пределом}, если для него выполняется уникальность, но не обязательно существование стрелки из определения предела.
Обратите внимание, что слабые пределы не обязательно уникальны.
Например, любой пулбэк $A \times_C B$ -- это слабый предел дискретной диаграммы, состоящей из объектов $A$ и $B$.

Докажите, что если для некоторой диаграммы существует предел $L$, то некоторый конус является слабым пределом тогда и только тогда, когда он является подобъектом $L$.

\begin{itemize}
\item[$\Leftarrow)$] Пусть $h: X \hookrightarrow L$. Скомпозируем эту стрелку со всеми стрелками $L$ и получим конус. Пусть есть другой конус $Y$ и существует морфизм $f: Y \to X$ такой, что следующая диаграмма коммутирует:
\[ \xymatrix{ 
Y \ar[r]^f \ar[d] & X \ar@{^{(}->}[d]^h \\
D(v)& L \ar[l]
} \]
Так как $L$ --- предел, то $h\circ f$ --- уникальный. Тогда если существует $f' : Y \to X$, для которого эта диаграмма тоже коммутирует, то  $h\circ f = h \circ f'$. Так как $h$ --- моно, то $f=f'$. Значит $X$ --- слабый предел.

\item[$\Rightarrow)$] Пусть $X$ --- слабый предел. Рассмотрим $h: X \to L$. Пусть для некоторых $f, g : Y \to X$ выполняется $h \circ f = h \circ g$. $Y$ является конусом, так как $X$ является конусом. Кроме того, $f = g$, так как $X$ --- слабый прелел. Значит $h$ --- моно.

\end{itemize}

\end{enumerate}

\end{document}
