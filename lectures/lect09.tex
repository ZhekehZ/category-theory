\documentclass{beamer}

\usepackage[english,russian]{babel}
\usepackage[utf8]{inputenc}
\usepackage[all]{xy}
\usepackage{ifthen}
\usepackage{xargs}

\usetheme{Szeged}
% \usetheme{Montpellier}
% \usetheme{Malmoe}
% \usetheme{Berkeley}
% \usetheme{Hannover}
\usecolortheme{beaver}

\newcommand{\newref}[4][]{
\ifthenelse{\equal{#1}{}}{\newtheorem{h#2}[hthm]{#4}}{\newtheorem{h#2}{#4}[#1]}
\expandafter\newcommand\csname r#2\endcsname[1]{#4~\ref{#2:##1}}
\newenvironmentx{#2}[2][1=,2=]{
\ifthenelse{\equal{##2}{}}{\begin{h#2}}{\begin{h#2}[##2]}
\ifthenelse{\equal{##1}{}}{}{\label{#2:##1}}
}{\end{h#2}}
}

\newref[section]{thm}{theorem}{Theorem}
\newref{lem}{lemma}{Lemma}
\newref{prop}{proposition}{Proposition}
\newref{cor}{corollary}{Corollary}

\theoremstyle{definition}
\newref{defn}{definition}{Definition}

\newcommand{\cat}[1]{\mathbf{#1}}
\renewcommand{\C}{\cat{C}}
\newcommand{\y}{\cat{y}}
\newcommand{\D}{\cat{D}}
\newcommand{\E}{\cat{E}}
\newcommand{\Set}{\cat{Set}}
\newcommand{\Grp}{\cat{Grp}}
\newcommand{\Mon}{\cat{Mon}}
\newcommand{\Ab}{\cat{Ab}}
\newcommand{\Ring}{\cat{Ring}}
\renewcommand{\Vec}{\cat{Vec}}
\newcommand{\Mat}{\cat{Mat}}
\newcommand{\Num}{\cat{Num}}

\newcommand{\Conus}{\mathrm{Conus}}
\newcommand{\Hom}{\mathrm{Hom}}
\newcommand{\limit}{\mathrm{lim}}
\newcommand{\colim}{\mathrm{colim}}

\newcommand{\pb}[1][dr]{\save*!/#1-1.2pc/#1:(-1,1)@^{|-}\restore}
\newcommand{\po}[1][dr]{\save*!/#1+1.2pc/#1:(1,-1)@^{|-}\restore}

\AtBeginSection[]
{
\begin{frame}[c,plain,noframenumbering]
\frametitle{План лекции}
\tableofcontents[currentsection]
\end{frame}
}

\makeatletter
\defbeamertemplate*{footline}{my theme}{
    \leavevmode
}
\makeatother

\begin{document}

\title{Теория категорий}
\subtitle{Категории предпучков}
\author{Валерий Исаев}
\maketitle

\section{$\Hom$-функторы}

\begin{frame}
\frametitle{Определение}
\begin{itemize}
\item Пусть $A$ -- объект некоторой категории $\C$. Тогда существуют функторы
\[ \Hom_\C(A, -) : \C \to \Set \]
\[ \Hom_\C(-, A) : \C^{op} \to \Set \]
\item На объектах они действуют следующим образом: $\Hom_\C(A, -)(B) = \Hom_\C(A, B)$ и $\Hom_\C(-, A)(B) = \Hom_\C(B, A)$.
\item На морфизмах они действуют следующим образом: $\Hom_\C(A, -)(f) = g \mapsto f \circ g$ и $\Hom_\C(-, A)(f) = g \mapsto g \circ f$.
\item Эти функторы называются (ковариантным и контравариантным) \emph{$\Hom$-функторами}.
\end{itemize}
\end{frame}

\begin{frame}
\frametitle{Определение (ко)пределов через $\Hom$-функторы}
\begin{itemize}
\item Существует определение (ко)пределов через $\Hom$-функторы и (ко)пределы в $\Set$.
\item Если $D : J \to \C$ -- диаграмма и $L$ -- объект $\C$, то пусть $\Conus_D(L) = \limit_{j \in J} \Hom_\C(L, D(j))$ -- множество конусов диаграммы $D$ с вершиной $L$.
\item Если $C$ -- конус диаграммы $D$, то существует естественное преобразование $\alpha_X : \Hom_\C(X,L) \to \Conus_D(X)$, определяемое как $\alpha_X(f)_j = C_j \circ f$.
\item Конус $C$ является пределом тогда и только тогда, когда $\alpha$ -- естественный изоморфизм.
\item Отсюда следует, что $\Hom(X,-)$ сохраняет пределы.
\end{itemize}
\end{frame}

\begin{frame}
\frametitle{Пример}
\begin{itemize}
\item Например, пусть $J$ -- дискретная категория на $\{ 1, 2 \}$, $D(j) = A_j$.
\item Тогда $L$ вместе с функиями $\pi_j : L \to A_j$ является произведением $A_1$ и $A_2$ тогда и только тогда, когда функции $\Hom(X,L) \to \Hom(X,A_1) \times \Hom(X,A_2)$,
порождаемые композицией с проекциями являются биекциями для любого $X$.
\item Аналогичные утверждения верны и для $\Hom(-, X) : \C^{op} \to \Set$. Например, он сохраняет пределы.
\item Действительно, $\Hom_\C(A \amalg B, X) \to \Hom_\C(A,X) \times \Hom_\C(B,X)$ является биекцией.
\end{itemize}
\end{frame}

\begin{frame}
\frametitle{Приложение}
\begin{itemize}
\item Докажем, что правый сопряженный функтор $G : \D \to \C$ сохраняет пределы.
\item Пусть $F$ -- левый сопряженный к $G$, $D : J \to \D$ -- диаграмма в $\D$ и $L$ -- ее предел. Тогда
\begin{align*}
\Hom(X,G(L)) & \simeq \\
\Hom(F(X),L) & \simeq \\
\limit_{j \in J} \Hom(F(X), D(j)) & \simeq \\
\limit_{j \in J} \Hom(X, GD(j)) & .
\end{align*}
\end{itemize}
\end{frame}

\section{Лемма Йонеды}

\begin{frame}
\frametitle{Лемма Йонеды}
\only<1>{
\begin{lem}
Пусть $a$ -- объект категории $\C$ и $F : \C^{op} \to \Set$ -- некоторый функтор.
Тогда существует биекция $\Hom_{\Set^{\C^{op}}}(\Hom_\C(-,a), F) \simeq F(a)$ естественная по $a$.
\end{lem}
}
\begin{proof}
\only<1>{
Если $\alpha : \Hom_\C(-,a) \to F$, то определим $T(\alpha) = \alpha_a(id_a) \in F(a)$.
Если $x \in F(a)$, то определим $S(x) : \Hom_\C(-,a) \to F$ как $S(x)_b(f) = F(f)(x)$.
Естественность $S(x)$ следует из того факта, что $F$ сохраняет композиции.
}
\only<2>{
Нужно проверить, что $T$ и $S$ взаимообратны.
Пусть $x \in F(a)$, тогда $T(S(x)) = S(x)_a(id_a) = F(id_a)(x) = x$.

Пусть $\alpha : \Hom_\C(-,a) \to F$.
Тогда $S(T(\alpha))_b(f) = F(f)(T(\alpha)) = F(f)(\alpha_a(id_a)) = \alpha_b(f)$.
Последнее равенство следует из естественности $\alpha$.

\medskip

Осталось проверить естественность по $a$.
Ее достаточно проверить для $S$.
Если $g : a \to c$ и $x \in F(c)$, то нужно проверить, что $S_c(F(g)(x))_b(f) = S_a(x)_b(g \circ f)$.
Это следует непосредственно из определения $S$.
}
\alt<2>{\qedhere}{\phantom\qedhere}
\end{proof}
\end{frame}

\begin{frame}
\frametitle{Вложение Йонеды}
\begin{itemize}
\item Для любой категории $\C$ функтор $\Hom_\C(-,-) : \C \to \Set^{\C^{op}}$ является полным и строгим.
\item Действительно, в лемме Йонеды достаточно взять в качестве $F$ функтор $\Hom_\C(-,b)$.
\item Этот функтор называется \emph{вложением Йонеды} и обозначается $\y : \C \to \Set^{\C^{op}}$.
\item Для того, чтобы проверить, что объекты $a$ и $b$ категории $\C$ изоморфны, достаточно проверить, что $\y a \simeq \y b$.
\item Если $f : a \to b$, то $f$ является изоморфизмом тогда и только тогда, когда для любого $c$ композиция с $f$ задает биекцию на множествах $\Hom_\C(c,a)$ и $\Hom_\C(c,b)$.
\end{itemize}
\end{frame}

\begin{frame}
\frametitle{Пример}
\begin{itemize}
\item В качестве примера использования этого факта покажем, что в декартово замкнутой категории верно $a^{b \amalg c} \simeq a^b \times a^c$.
\item Действительно,
\begin{align*}
\Hom(x,a^{b \amalg c}) & \simeq \\
\Hom(b \amalg c,a^x) & \simeq \\
\Hom(b,a^x) \times \Hom(c,a^x) & \simeq \\
\Hom(x,a^b) \times \Hom(x,a^c) & \simeq \\
\Hom(x,a^b \times a^c) & .
\end{align*}
\end{itemize}
\end{frame}

\begin{frame}
\frametitle{Ко- лемма Йонеды}
\begin{itemize}
\item Интуитивно, ко- лемма Йонеды говорит, что произвольный функтор $P : \C^{op} \to \Set$ является копределом функторов вида $\y a$.
\item Объекты категории $J_P$ -- это пары $(a,x)$, где $a$ -- объект $\C$, $x \in P(a)$.
\item Морфизмы между $(a,x)$ и $(b,y)$ -- это морфизмы $f : a \to b$ категории $\C$, такие, что $P(f)(y) = x$.
\item Диаграмма $D_P : J_P \to \Set^{\C^{op}}$ каждому $(a,x)$ сопоставляет $\y a$.
\item \emph{Ко- лемма Йонеды} утверждает, что $P$ является копределом $D_P$.
\end{itemize}
\end{frame}

\begin{frame}
\frametitle{Ко- лемма Йонеды}
\begin{proof}
\only<1>{
По лемме Йонеды достаточно проверить, что $\Hom(P, -) \simeq \Hom(\colim\,D_P, -)$.
У нас есть следующая последовательность биекций.
\begin{align*}
\Hom(\colim_{(a,x) \in J}(\y a), R) & \simeq \\
\limit_{(a,x) \in J} \Hom(\y a, R) & \simeq \\
\limit_{(a,x) \in J} R_a & .
\end{align*}
}

\only<2>{
Нужно проверить, что $\Hom(P,R) \simeq \limit_{(a,x) \in J} R_a$, и эта биекция естественна по $R$.

Но элементы множества $\Hom(P,R)$ -- это естественные преобразования, то есть функции, которые каждому $a \in Ob(\C)$ и $x \in P_a$ сопоставляют элемент $R_a$ и удовлетворяют условию естественности.

С другой стороны, множество $\limit_{(a,x) \in J} R_a$ состоит из таких же функций $\alpha$, удовлетворяющих условию,
что для любого $f : a \to b$ если $P(f)(y) = x$, то  $\alpha(a,x) = R(f)(\alpha(b,y))$.
Это условие в точности условие естественности $\alpha$.

Естественность по $R$ проверяется напрямую.
}
\alt<2>{\qedhere}{\phantom\qedhere}
\end{proof}
\end{frame}

\section{Категории предпучков}

\begin{frame}
\frametitle{Примеры}
\begin{itemize}
\item \emph{Предпучок} на малой категории $\C$ -- это функтор $\C^{op} \to \Set$. Категория предпучков -- это категория функторов $\Set^{\C^{op}}$.
\item Если $P$ -- предпучок на $\C$ и $a$ -- объект $\C$, то мы будем писать $P_a$ вместо $P(a)$.
\item Категория графов -- это категория предпучков.
\item Категория последовательностей множеств
\[ \xymatrix{X_1 \ar[r] & X_2 \ar[r] & X_3 \ar[r] & \ldots } \]
является категорией предпучков.
\item Категория действий группы $G$ -- категория предпучков на $G$.
\end{itemize}
\end{frame}

\begin{frame}
\frametitle{Свойства категорий предпучков}
\begin{itemize}
\item Категории предпучков полные и кополные.
\item Действительно, в категориях функторов пределы и копределы считаются поточечно.
\item Категории предпучков декартовы замкнуты.
\item Действительно, пусть $P$, $R$ -- предпучки на $\C$.
Если $R^P$ существует, то для любого $a \in \C$ должно быть верно
\[ (R^P)_a \simeq \Hom(\y a, R^P) \simeq \Hom(\y a \times P, R) \]
\item Мы можем использовать это свойство как определение $R^P$.
\end{itemize}
\end{frame}

\begin{frame}
\frametitle{Доказательство}
Мы уже видели, что свойство экспоненты верно для представимых функторов.
Осталось проверить для произвольных:
\begin{align*}
\Hom(X,R^P) & \simeq \\
\Hom(\colim_a(\y a),R^P) & \simeq \\
\limit_a(\Hom(\y a,R^P)) & \simeq \\
\limit_a(\Hom(\y a \times P, R)) & \simeq \\
\Hom(\colim_a(\y a \times P), R) & \simeq \\
\Hom(\colim_a(\y a) \times P, R) & \simeq \\
\Hom(X \times P, R) &
\end{align*}
\end{frame}

\begin{frame}
\frametitle{Генераторы категории}
\begin{itemize}
\item Коллекция $S$ объектов категории $\C$ называется ее \emph{генератором}, если для любой пары морфизмов $f,g : A \to B$ верно, что
если $f \circ s = g \circ s$ для любого морфизма $s$ с доменом в $S$, то $f$ = $g$.
\item Другими словами, чтобы проверить равенство двух стрелок, достаточно проверить их равенство на объектах из $S$.
\item Если $S$ является генератором категории $\C$ и $x \in S$, то морфизмы вида $x \to a$ называют \emph{обобщенными элементами} объекта $a$.
\end{itemize}
\end{frame}

\begin{frame}
\frametitle{Примеры генераторов}
\begin{itemize}
\item В $\Set$ генератором является множество, состоящее из одного одноэлементного множества.
\item В $\Grp$ генератором является $\mathbb{Z}$.
\item Обощенные элементы для этого генератора -- это просто элементы множества.
\item Коллекция объектов вида $\y a$ является генератором для категории предпучков.
\end{itemize}
\end{frame}

\section{Примеры}

\subsection{Категория графов}

\begin{frame}
\frametitle{Категория графов $\cat{Graph}$}
\begin{itemize}
\item В этом разделе мы просто разберем подробно пример категории графов.
\item Граф $G$ состоит из множества вершин $G_V$, множества ребер $G_E$ и пары функций $d,c : G_E \to G_V$, сопостовляющих каждому ребру его начало и конец.
\item Таким образом, категория графов $\cat{Graph}$ -- это категория предпучков на категории $\C$, состоящей из двух объектов $V$ и $E$ и двух не тождественных морфизмов.
\[ \xymatrix{ V \ar@<-.5ex>[r]_c \ar@<.5ex>[r]^d & E } \]
\end{itemize}
\end{frame}

\begin{frame}
\frametitle{Лемма Йонеды для $\cat{Graph}$}
\begin{itemize}
\item Вложение Йонеды говорит, что категория $\C$ вкладывается в категорию графов.
\item $\y V$ -- это граф, состоящий из одной вершины.
\item $\y E$ -- это граф, состоящий из одного ребра.
\item $\y d$ и $\y c$ -- функции, отображающие единственную вершину графа $\y V$ в левый или правый конец единственного ребра графа $\y E$.
\item Лемма Йонеды говорит, что $\Hom_{\cat{Graph}}(\y V, G) \simeq G_V$.
\item Действительно, морфизмы из графа $\y V$ в $G$ -- это в точности вершины графа $G$.
\item Лемма Йонеды говорит, что $\Hom_{\cat{Graph}}(\y E, G) \simeq G_E$.
\item Действительно, морфизмы из графа $\y E$ в $G$ -- это в точности ребра графа $G$.
\end{itemize}
\end{frame}

\begin{frame}
\frametitle{Ко- лемма Йонеды для $\cat{Graph}$}
\begin{itemize}
\item О копределах можно думать геометрически.
\item Например, если $G_1$ -- подграф графов $G_2$ и $G_3$, то пушаут $G_2 \amalg_{G_1} G_3$ -- это граф, ``склеенный'' из $G_2$ и $G_3$ вдоль $G_1$.
\item Таким, образом ко- лемма Йонеды говорит, что любой предпучок можно ``склеить'' из представимых предпучков.
\item В случае с графами это означает, что любой граф можно склеить из графов $\y V$ и $\y E$.
\item Действительно, любой граф можно склеить из вершин и ребер, которые в нем содержатся.
\end{itemize}
\end{frame}

\begin{frame}
\frametitle{Генераторы в категории графов}
\begin{itemize}
\item Коллекция графов $\{ \y V, \y E \}$ является генератором категории $\cat{Graph}$.
\item Действительно, чтобы проверить, что морфизмы графов $f,g : G \to H$ равны, достаточно проверить, что они совпадают на вершинах и на ребрах.
\item Другими словами, достаточно проверить, что они совпадают на морфизмах из $\y V$ и $\y E$.
\item Обобщенный элемент для этого генератора -- это либо вершина графа (обобщенный элемент вида $\y V$), либо его ребро (обощенный элемент вида $\y E$).
\end{itemize}
\end{frame}

\subsection{Категория глобулярных множеств}

\begin{frame}
\frametitle{Категория глобулярных множеств $\cat{Glob}$}
\begin{itemize}
\item Глобулярное множество -- это предпучок на следующей категории:
\[ \xymatrix{ 0 \ar@<-.5ex>[r]_{t_0} \ar@<.5ex>[r]^{s_0} & 1 \ar@<-.5ex>[r]_{t_1} \ar@<.5ex>[r]^{s_1} & 2 \ar@<-.5ex>[r]_{t_2} \ar@<.5ex>[r]^{s_2} & \ldots } \]
где $s_{i+1} \circ s_i = t_{i+1} \circ s_i$ и $s_{i+1} \circ t_i = t_{i+1} \circ t_i$.
\item Другими словами, объекты этой категории -- натуральные числа, $\Hom(n,n) = \{ id_n \}$, $\Hom(n,k) = \varnothing$ если $n > k$, и $\Hom(n,k)$ состоит из двух элементов, если $n < k$.
\end{itemize}
\end{frame}

\begin{frame}
\frametitle{Лемма Йонеды для $\cat{Glob}$}
\begin{itemize}
\item О глобулярных множествах вида $\y n$ можно думать как о $n$-мерных шарах.
\item $\y 0$ -- точки, $\y 1$ -- отрезки, $\y 2$ -- диски, $\y 3$ -- 3-мерные шары, и т.д.
\item $\y s_n$ вкладывает $n$-мерный шар в верхнюю половну границы $(n+1)$-мерного шара.
\item $\y t_n$ вкладывает $n$-мерный шар в нижнюю половну границы $(n+1)$-мерного шара.
\item Таким образом, $\y s_n$ и $\y t_n$ пересекаются по $(n-1)$-мерным шарам.
\item Лемма Йонеды говорит, что если $X$ -- глобулярное множество, то $\Hom(\y n, X)$ изоморфно множеству $X_n$ его $n$-мерных шаров.
\item Ко- лемма Йонеды говорит, что любое глобулярное множество склеено из шаров.
\end{itemize}
\end{frame}

\end{document}
